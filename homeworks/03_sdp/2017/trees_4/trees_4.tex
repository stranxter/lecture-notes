\documentclass[12pt,a4paper]{article}
\usepackage[utf8]{inputenc}
\usepackage{amsmath}
\usepackage{amsfonts}
\usepackage{amssymb}
\usepackage{listings}
\usepackage{url}
\usepackage[bulgarian]{babel}
\usepackage{listings}
\usepackage{enumerate}
\usepackage{hyperref}
\usepackage{relsize}
\usepackage{graphicx}


\lstset{breaklines=true}


\author{\textit{email: kalin@fmi.uni-sofia.bg}}
\title{\textsc{Задачи за задължителна самоподготовка} \\
по \\
Структури от данни и програмиране\\
\textit{Двоични дървета, стек, итератори}}




\begin{document}
\maketitle



\begin{enumerate}

	\item Разработеният на лекции итератор за шаблон \texttt{BTree<T>} да бъде видоизменен така, че са се позволи конфигуриране на типа на обхождането при създаване на итератора. Да се реализират следните обхождания:

	\begin{enumerate}
		\item КДЛ и ЛДК.
    \item Обхождане в широчина.
		\item Обхождане само листата на дървото, от ляво надясно.
		\item Обхождане само на тези елементи на дървото (при коя да е от горните последователности), за които е удовлетворен даден предикат \texttt{bool pred (const T\&)}.
  \end{enumerate}


\end{enumerate}






\end{document}
