\documentclass[12pt,a4paper]{article}
\usepackage[utf8]{inputenc}
\usepackage{amsmath}
\usepackage{amsfonts}
\usepackage{amssymb}
\usepackage{listings}
\usepackage{url}
\usepackage[bulgarian]{babel}
\usepackage{listings}
\usepackage{enumerate}
\usepackage{hyperref}
\usepackage[framemethod=tikz]{mdframed}
\usepackage{relsize}


\newcommand{\code}[1]{\texttt{#1}}

\lstset{breaklines=true}


\author{\textit{email: kalin@fmi.uni-sofia.bg}}
\title{\textsc{Задачи за задължителна самоподготовка} \\
по \\
Структури от данни и програмиране\\
\textit{Линеен едносвързан списък}}



\begin{document}
\maketitle


\begin{enumerate}

	\item  Да се решат следните задачи от ЗЗС #1 чрез използване на итератор за списък: задачи 1, 6, 8, 10, 11. Т.е., да се дефинират функции, които получават итератор и извършват съответните проверки.
	\item Да се тестват същите функции с итератор на масив.

	\item \code{IteratorBase} и производните му да се обогатят с методи \code{getPrevious()} и съответно \code{hasPrevious()}, които навигират итераторите към предишния елемент на структурата от данни.

\end{enumerate}


	\vspace{20px}


\end{document}
