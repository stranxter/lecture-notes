\documentclass[12pt,a4paper]{article}
\usepackage[utf8]{inputenc}
\usepackage{amsmath}
\usepackage{amsfonts}
\usepackage{amssymb}
\usepackage{listings}
\usepackage{url}
\usepackage[bulgarian]{babel}
\usepackage{listings}
\usepackage{enumerate}
\usepackage{hyperref}
\usepackage[framemethod=tikz]{mdframed}
\usepackage{relsize}


\newcommand{\code}[1]{\texttt{#1}}

\lstset{breaklines=true}


\author{\textit{email: kalin@fmi.uni-sofia.bg}}
\title{\textsc{Задачи за задължителна самоподготовка} \\
по \\
Структури от данни и програмиране\\
\textit{``Реактивни'' безкрайни потоци}}



\begin{document}
\maketitle


ВНИМАНИЕ: Решението на следите задачи е публикувано към кода от лекции. Разглeдайте решенията, осмислете ги и ги пресъздайте самостоятелно (или предложете друго решение на задачата за реализация на операции над безкрайни потоци).

\begin{enumerate}

	\item  Към класа \texttt{StreamBase} от реализираната на лекции йерархия да се добави метод за печатане на първите \texttt{n} елемента от потока. Методът да връща остатъка от потока.


	Следният пример отпечатва първите 5 нечетни числа:
	\begin{verbatim}
			ints.filter(odd).print(2).print(3);
	\end{verbatim}

	\item Да се дефинира поток \texttt{RepeatStream}, състоящ се от безкрайно повторение на дадено число.

	Следният пример отпечатва 5 единици:
	\begin{verbatim}
RepeatStream ones(1);
ones.print(5);
	\end{verbatim}


	\item Да се дефинира клас \texttt{SumStream}, който позволява по дадени два потока \texttt{A} и \texttt{B} да се генерира нов поток, всеки от елементите на който е сумата (получена с оператора \texttt{+}) на двата съответни елемента на \texttt{A} и \texttt{B}.

	Следният пример отпечатва първите 5 четни числа:
	\begin{verbatim}
			ints.sum(ints).print (5);
	\end{verbatim}

	Следният пример отпечатва 5 двойки:
	\begin{verbatim}
			ones.sum(ones).print (5);
	\end{verbatim}


	\item Да се дефинира клас \texttt{ZipStream}, който позволява по дадени два потока $A=(a_i)$ и $B=(b_i)$ да се генерира нов поток $C=(c_i)$, всеки от елементите на който е резултат от приложението на някаква функция $f:int \times int \rightarrow int$ над двата съответни елемента на \texttt{A} и \texttt{B}, т.е. $c_i=f(a_i,b_i)$.

	Следният пример отпечатва първите 5 точни квадрата:
	\begin{verbatim}
			ints.zip(ints,mult).print (5);
	\end{verbatim}
	 Където $mult(x,y)=x*y$

	 \item Какво трябва да се промени в йерархията така, че да може да се конструират потоци с елементи, различни от $int$?

\end{enumerate}


	\vspace{20px}


\end{document}
