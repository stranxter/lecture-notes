\documentclass[12pt,a4paper]{article}
\usepackage[utf8]{inputenc}
\usepackage{amsmath}
\usepackage{amsfonts}
\usepackage{amssymb}
\usepackage{listings}
\usepackage{url}
\usepackage[bulgarian]{babel}
\usepackage{listings}
\usepackage{enumerate}
\usepackage{hyperref}
\usepackage[framemethod=tikz]{mdframed}
\usepackage{relsize}
\usepackage{enumitem}


\newcommand{\code}[1]{\texttt{#1}}

\lstset{breaklines=true}


\author{\textit{email: kalin@fmi.uni-sofia.bg}}
\title{\textsc{Задачи за задължителна самоподготовка} \\
по \\
Структури от данни и програмиране\\
\textit{Контейнери и итератори}}



\begin{document}
\maketitle

Следните задачи да се решат като упражнение за директно боравене с възлите на линеен двусвързан списък. Функциите да се тестват с подходящи тестове.

\begin{enumerate}

	\item  Да се дефинира функция \code{int count(dllnode<T>* l,int x)}, която преброява колко пъти елементът \code{x} се среща в списъка с първи елемент \code{l}.
	\item  Фунцкция \code{dllnode<int>* range (int x, int y)} която създава и връща първия елемент на списък с елементи $x, x+1, ..., y$, при положение, че $x \leq y$.
	\item  Да се дефинира функция \code{removeAll (dllnode<T>*\& l,const T\& x)}, която изтрива всички срещания на елемента \code{x} от списъка \code{l}.
	\item  Да се дефинира функция \code{void append(dllnode*<T>\& l1, dllnode<T>* l2)}, която добавя към края на списъка $l_1$ всички елементи на списъка $l_2$. Да се реализира съответен оператор \texttt{+=} в класа на списъка.
	\item  Да се дефинира функция \code{dllnode* concat(dllnode<T>* l1, dllnode<T>* l2)}, който съединява два списъка в нов, трети списък. Т.е. \code{concat($l_1,l_2$)} създава и връща нов списък от елементите на \code{$l_1$}, следвани от елементите на \code{$l_2$}. Да се реализира съответен оператор \texttt{+} в класа на списъка.
	\item  Да се дефинира функция \code{reverse}, която обръща реда на елементите на списък. Например, списъкът с елементи $1,2,3$ ще се преобразува до списъка с елементи $3,2,1$.
	\item Да се напише функция \code{void removeduplicates (dllnode *\&l)}, която изтрива всички дублиращи се елементи от списъка $l$.
\end{enumerate}

Следните задачи да се решат като упражнение за работа с итератори. Функциите да се тестват с подходящи тестове върху двата вида контейнери. Има ли разлика в проиводителността за някои от тях?

\begin{enumerate}[resume]

	\item Да се разшири интераторът на динамичния масив така, че да поддържа оператора за стъпка назад \texttt{-{}-}.

	\item  Да се дефинира функция \code{map}, която прилага едноаргументна функция $f:int \rightarrow int$ към всеки от елементите на произволен контейнер. Да се дефнира и шаблон на функцията за списък с произолен тип на елементите.

	\item Да се напише функция \code{bool duplicates (...)}, която проверява дали в контейнер има дублиращи се елементи.

	\item Да се напише фунцкия \code{bool issorted (...)}, която проверява дали елементите на даден контейнер са подредени в нарастващ или в намаляващ ред.

	\item Да се напише фунцкия \code{bool palindrom (...)}, която проверява дали редицата от елементите на даден контейнер обрзува палиндром (т.е. дали се чете еднакво както отляво надясно така и отдяно наляво).

\end{enumerate}


	\vspace{20px}


\end{document}
