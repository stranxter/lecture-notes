\documentclass[12pt,a4paper]{article}
\usepackage[utf8]{inputenc}
\usepackage{amsmath}
\usepackage{amsfonts}
\usepackage{amssymb}
\usepackage{listings}
\usepackage{url}
\usepackage[bulgarian]{babel}
\usepackage{listings}
\usepackage{enumerate}
\usepackage{hyperref}
\usepackage{relsize}

\lstset{breaklines=true} 


\author{\textit{email: kalin@fmi.uni-sofia.bg}}
\title{\textsc{Задачи за задължителна самоподготовка} \\
по \\
Структури от данни и програмиране\\
\textit{Задачи върху интерпретатора}}



\begin{document}
\maketitle

Задачи върху разработвания на лекции интерпретатор:

\begin{enumerate}
	\item Да се измени интерпретаторът така, че да може при извикване на функция да се подават повече параметри, отколкото са необходими. Например фунцията \texttt{recprint} да може да се изпълнява с

	 \texttt{recprint 10 10 10 .}


	\item Да реализира възможност функциите да достъпват глобалните променливи. Да се приеме, че глобалните променливи са тези, които са в нулевата стекова рамка.

	\item Да се добави възможност функциите да се предават като параметри и да се връщат като стойности. 

	\begin{flushleft}
	\relscale{0.6} 
	Упътване: 
	\begin{itemize}
		\item Нужен е нов наследник на \texttt{Value}, който представя ``стойност-функция''. 
		\item \texttt{define} трябва да записва дефинираната-функция стойност в стековата рамка, заедно с останалите променливи. 

		Най-добре би било \texttt{define} да се елеминира и да се замести с \texttt{lambda}, която само създава стойност-функция, а самото записване в стековата рамка да става с \texttt{assign}. Например:

		\texttt{assign sum lambda a b c do + a + b c}


		\item \texttt{call} трябва да се промени така, че вместо име на функция (идентификатор), да получава израз, който да се предполага, че се оценява до стойност-функция. 
	\end{itemize}
		
	\end{flushleft}
\end{enumerate}






\end{document}

