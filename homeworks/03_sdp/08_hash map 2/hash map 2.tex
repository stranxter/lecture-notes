\documentclass[12pt,a4paper]{article}
\usepackage[utf8]{inputenc}
\usepackage{amsmath}
\usepackage{amsfonts}
\usepackage{amssymb}
\usepackage{listings}
\usepackage{url}
\usepackage[bulgarian]{babel}
\usepackage{listings}


\lstset{breaklines=true} 


\author{\textit{email: kalin@fmi.uni-sofia.bg}}
\title{\textsc{Задачи за задължителна самоподготовка} \\
по \\
Структури от данни и програмиране}



\begin{document}
\maketitle


\begin{enumerate}

	\item Да се дефинира \texttt{operator *} на шаблона на хеш-таблицата. Хеш-таблицата \texttt{c}, която се получава при \texttt{c = a * b}, да съдържа като множество от ключове сечението на множествата на ключове на \texttt{a} и \texttt{b}, като стойноста на всеки ключ е двойка (\texttt{std::pair}) от съответните стойности от \texttt{a} и \texttt{b}. Хеш-функцията на \texttt{c} да е същата като на \texttt{b}.

	\item Да се дефинира \texttt{operator +} на шаблона на хеш-таблицата. Хеш-таблицата \texttt{c}, която се получава при \texttt{c = a + b}, да съдържа като ключове симетричната разлика на ключовете на \texttt{a} и \texttt{b}, със съответните им стойности от \texttt{a} и \texttt{b}. Хеш-функцията на \texttt{c} да е същата като на \texttt{b}.

	\emph{Симетрична разлика на множествата $A$ и $B$ наричаме множеството $C = A \Delta B = A \cup B - A \cap B$, съдържащо тези елементи на $A$, които не са елементи на $B$ и тези елементи на $B$, които не са елементи на $A$.}

	\item Да се дефинира метод 

	\texttt{void map (void (*f) (ValueType\&))} 

	на хеш-таблицата, който прилага функцията \texttt{f} над всички стойности в хеш-таблицата.

	\item Да се дефинира метод 

	\texttt{void mapKeys (KeyType (*f) (const KeyType\&))} 

	на хеш-таблицата, който замества всеки ключ \texttt{key} на хеш-таблицата с \texttt{f(key)}, като се запазва старата му стойност. 

	\emph{Упътване: Да се извърши съответното ре-хеширане на елемента и той да се премести на съответния нов индекс в таблицата.}


	\item Методът \texttt{begin} на хеш-таблицата да се допълни така, че да може да получава и предикат $p:KeyType \rightarrow bool$. Резултатният итератор да итерира само през тези ключове от таблицата, които удовлетворяват $p$.


\end{enumerate}




\end{document}

