\documentclass[12pt,a4paper]{article}
\usepackage[utf8]{inputenc}
\usepackage{amsmath}
\usepackage{amsfonts}
\usepackage{amssymb}
\usepackage{listings}
\usepackage{url}
\usepackage[bulgarian]{babel}
\usepackage{listings}
\usepackage{enumerate}
\usepackage[framemethod=tikz]{mdframed}


\lstset{breaklines=true}


\author{\textit{email: kalin@fmi.uni-sofia.bg}}
\title{\textsc{Задачи за задължителна самоподготовка} \\
по \\
Увод в програмирането\\
\textit{Машини с неограничени регистри}}



\begin{document}
\maketitle



\small{Дефиницията на Машина с неогрничени регистри по-долу е взаимствана от учебника \textit{А. Дичев, И. Сосков, ``Теория на програмите'', Издателство на СУ, София, 1998}.

\vspace{20px}

\begin{mdframed}[hidealllines=true,backgroundcolor=gray!20]

	``Машина с неограничени регистри'' (или МНР) наричаме абстрактна машина, разполагаща с неограничена памет. Паметта на машината се представя с безкрайна редица от естествени числа $m[0],m[1],...$, където $m[i] \in \mathcal{N}$. Елементите $m[i]$ на редицата наричаме ``клетки'' на паметта на машината, а числото $i$ наричаме ``адрес'' на клетката $m[i]$.

	 МНР разполага с набор от инструцкии за работа с паметта. Всяка инструкция получава един или повече параметри (операнди) и може да предизвика промяна в стойността на някоя от клетките на паметта. Инструкциите на МНР за работа с паметта са:

	\begin{enumerate}
		\item \texttt{ZERO n}: Записва стойността 0 в клетката с адрес $n$
		\item \texttt{INC n}: Увеличава с единица стойността, записана в клетката с адрес $n$
		\item \texttt{MOV x y}: Присвоява на клетката с адрес $y$ стойността на клетката с адрес $x$
	\end{enumerate}

	``Програма'' за МНР наричаме всяка последователност от инструкции на МНР и съответните им операнди. Всяка инструкция от програмата индексираме с поредния ѝ номер. Изпълнението на програмата започва от първата инструкция и преминава през всички инструкции последователно, освен в някои случаи, опиани по-долу. Изпълнението на програмата се прекратвя след изпълнението на последната ѝ инструкция. Например, след изпълнението на следната програма:

	\begin{verbatim}
	0: ZERO 0
	1: ZERO 1
	2: ZERO 2
	3: INC 1
	4: INC 2
	5: INC 2
	\end{verbatim}

	Първите три клетки на машината ще имат стойност 0, 1, 2, независимо от началните им стойности.

	Освен инструкциите за работа с паметта, МНР притежават и една инструкция за промяна на последователноста на изпълнение на програмата:

	\begin{enumerate}
	\setcounter{enumi}{4}
		\item \texttt{JUMP x}: Изпълнението на програмата ``прескача'' и продължава от инструкцията с пореден номер $x$. Ако програмата има по-малко от $x+1$ инструкции, изпълнението ѝ се прекратява
		\item \texttt{JUMP x y z}: Ако съдържанията на клетките  $x$ и $y$ съвпадат, изпълнението на програмата ``прескача'' и продължава от инструкцията с пореден номер $z$. В противен случай, програмата продължава със следващата инструкция. Ако програмата има по-малко от $z+1$ инструкции, изпълнението ѝ се прекратява
	\end{enumerate}

	Например, нека изпълнето на следната програма започва при стойности на клиетките на паметта 10,0,0,...:

	\begin{verbatim}
	0: JUMP 0 1 5
	1: INC 1
	2: INC 2
	3: INC 2
	4: JUMP 0

	\end{verbatim}

	След приключване на програмата, първите три клетки на машината ще имат стойности 10, 10, 20.

\end{mdframed}

Задачи:

\begin{enumerate}
	\item Нека паметта на МНР е инициалирана с редицата $m,n,0,0,...$. Да се напише програма на МНР, след изпълнението на която клетката с адрес 2 съдържа числото $m+n$.
	\item Нека паметта на МНР е инициалирана с редицата $m,n,0,0,...$. Да се напише програма на МНР, след изпълнението на която клетката с адрес 2 съдържа числото $m \times n$.
	\item Нека паметта на МНР е инициалирана с редицата $m,n,0,0,...$. Да се напише програма на МНР, след изпълнението на която клетката с адрес 2 съдържа числото 1 тогава и само тогава, когато $m>n$ и числото 0 във всички останали случаи.
\end{enumerate}




\end{document}
