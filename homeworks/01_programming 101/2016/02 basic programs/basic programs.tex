\documentclass[12pt,a4paper]{article}
\usepackage[utf8]{inputenc}
\usepackage{amsmath}
\usepackage{amsfonts}
\usepackage{amssymb}
\usepackage{listings}
\usepackage{url}
\usepackage[bulgarian]{babel}
\usepackage{listings}
\usepackage{enumerate}


\lstset{breaklines=true} 


\author{\textit{email: kalin@fmi.uni-sofia.bg}}
\title{\textsc{Задачи за задължителна самоподготовка} \\
по \\
Увод в програмирането\\
\textit{базови програми}}



\begin{document}
\maketitle


\begin{enumerate}


	\item Превърнете рожденната си дата шестнадесетична, в осмична и в двоична бройни системи.

	\item Разгледайте стандартната ASCii таблица (\texttt{http://www.asciitable.com/}) и запишете името си чрез серия от ASCii кодове. 

	\item Задача 1.20.

	Да се напише програма, която по въведени от клавиатурата цели числа x и k ($k \geq 1$) намира и извежда на екрана k-тата цифра на х. Броенето да е отдясно наляво.

	\item Задача 2.7.

	Да се напише програма, която въвежда координатите на точка от равнина и извежда на кой квадрант принадлежи тя. Да се разгледат случаите, когато точката принадлежи на някоя от координатните оси или съвпада с центъра на координатната система.

	\item Задача 2.12.

	Да се напише програма, която проверява дали дадена година е високосна.

	\item Задача 2.40.

	Да се напише програма, която (чрез цикъл for) намира сумата на всяко трето цяло число, започвайки от 2 и ненадминавайки n (т.е. сумата 2 + 5 + 8 + 11 + ...).

	\item Задача 2.44.

	Дадено е естествено число n ($n \geq 1$). Да се напише програма, която намира броя на тези елементи от серията числа $i^3 + 13 \times i \times n + n
	^3$ , $i = 1, 2, ..., n$, които са кратни на 5 или на 9.

	\vspace{20px}

	\small{Задачите са от сборника \textit{Магдалина Тодорова, Петър Армянов, Дафина Петкова, Калин Николов, ``Сборник от задачи по програмиране на C++. Първа част. Увод в програмирането''}. Запазена е номерацията в сборника.}

\end{enumerate}




\end{document}

