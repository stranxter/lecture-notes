\clearpage\section {Линеен едносвързан и двусвързан списък}

\subsection {Представяне на двусвързан списък}

\begin{mdframed}[hidealllines=true,backgroundcolor=gray!20]
Възел на линеен двусвързан списък представяме със следния шаблон на структура:
\begin{verbatim}
template <class T>
struct dllnode
{
  T data;
  dllnode<T> *next, *previous;
};
\end{verbatim}
Освен ако не е указано друго, задачите по-долу да се решат като се реализират методи на клас \code{DLList} със следния скелет:
\begin{verbatim}
template <class T>
class DLList
{
  //...
  private:
  dllnode<T> *first, *last;
};
\end{verbatim}
Преди да пристъпите към задачите, реализирайте подходящи контруктори, деструктор и оператор за присвояване на класа.
\end{mdframed}

\begin{figure}
    \centering
  
      \begin{tikzpicture}[auto, node distance=2cm,>=latex']
      \node[lnode] (n1) {\nodepart{two}1};
      \node[lnode, right of = n1] (n2) {\nodepart{two}2};
      \node[lnode, right of = n2] (n3) {\nodepart{two}3};
      \node[lnode, right of = n3] (n4) {\nodepart{two}4};
      \node[lnode, right of = n4] (n5) {\nodepart{two}5};
  
      \node[rectangle,left of = n1](start){};
  
      \draw[*->]  (start)-- (n1);
  
      \draw[*->] let \p1 = (n2.three), \p2 = (n1.center) in (\x1,\y2) -- (n3);
      \draw[*->] let \p1 = (n1.three), \p2 = (n1.center) in (\x1,\y2) -- (n2);
      \draw[*->] let \p1 = (n3.three), \p2 = (n1.center) in (\x1,\y2) -- (n4);
      \draw[*->] let \p1 = (n4.three), \p2 = (n1.center) in (\x1,\y2) -- (n5);
      \draw[*->,dashed] let \p1 = (n2.one), \p2 = (n1.center) in ([shift={(0.1cm,-0.1cm)}]\x1,\y2) |-([shift={(0,0.5cm)}]n2.north west) -- ([shift={(0,0.5cm)}]n1.north) -| ([shift={(-1cm,0cm)}]n1);
      \draw[*->,dashed] let \p1 = (n3.one), \p2 = (n2.center) in ([shift={(0.1cm,0.1cm)}]\x1,\y2) |-([shift={(0,-1cm)}]n3.north west) -- ([shift={(0,-1cm)}]n2.north) -| ([shift={(-1cm,0cm)}]n2);
      \draw[*->,dashed] let \p1 = (n4.one), \p2 = (n3.center) in ([shift={(0.1cm,-0.1cm)}]\x1,\y2) |-([shift={(0,0.5cm)}]n4.north west) -- ([shift={(0,0.5cm)}]n3.north) -| ([shift={(-1cm,0cm)}]n3);
      \draw[*->,dashed] let \p1 = (n5.one), \p2 = (n4.center) in ([shift={(0.1cm,0.1cm)}]\x1,\y2) |-([shift={(0,-1cm)}]n5.north west) -- ([shift={(0,-1cm)}]n4.north) -| ([shift={(-1cm,0cm)}]n4);
      \end{tikzpicture}
    \caption{Двусвързан списък}
    \label{fig:skiplist}
  \end{figure}
  

Следните задачи да се решат като упражнение за директно боравене с възлите на линеен двусвързан списък. Функциите (методите) да се тестват с подходящи тестове.

\begin{enumerate}

	\item  Да се дефинира функция \code{int count(dllnode<T>* l,int x)}, която преброява колко пъти елементът \code{x} се среща в списъка с първи елемент \code{l}.
	\item  Фунцкция \code{dllnode<int>* range (int x, int y)} която създава и връща първия елемент на списък с елементи $x, x+1, ..., y$, при положение, че $x \leq y$.
	\item  Да се дефинира функция \code{removeAll (dllnode<T>*\& l,const T\& x)}, която изтрива всички срещания на елемента \code{x} от списъка \code{l}.
	\item  Да се дефинира функция \code{void append(dllnode*<T>\& l1, dllnode<T>* l2)}, която добавя към края на списъка $l_1$ всички елементи на списъка $l_2$. Да се реализира съответен оператор \texttt{+=} в класа на списъка.
	\item  Да се дефинира функция \code{dllnode* concat(dllnode<T>* l1, dllnode<T>* l2)}, който съединява два списъка в нов, трети списък. Т.е. \code{concat($l_1,l_2$)} създава и връща нов списък от елементите на \code{$l_1$}, следвани от елементите на \code{$l_2$}. Да се реализира съответен оператор \texttt{+} в класа на списъка.
	\item  Да се дефинира функция \code{reverse}, която обръща реда на елементите на списък. Например, списъкът с елементи $1,2,3$ ще се преобразува до списъка с елементи $3,2,1$.
	\item Да се напише функция \code{void removeduplicates (dllnode *\&l)}, която изтрива всички дублиращи се елементи от списъка $l$.
\end{enumerate}
