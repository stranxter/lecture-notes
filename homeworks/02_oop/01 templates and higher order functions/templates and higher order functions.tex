\documentclass[12pt,a4paper]{article}
\usepackage[utf8]{inputenc}
\usepackage{amsmath}
\usepackage{amsfonts}
\usepackage{amssymb}
\usepackage{listings}
\usepackage{url}
\usepackage[bulgarian]{babel}
\usepackage{listings}
\usepackage{enumerate}


\lstset{breaklines=true} 


\author{\textit{email: kalin@fmi.uni-sofia.bg}}
\title{\textsc{Задачи за задължителна самоподготовка} \\
по \\
Обектно-ориентирано програмиране\\
\textit{Шаблони и указатели към функции}}



\begin{document}
\maketitle


\begin{enumerate}
	\item Да се реализира шаблон на функция \texttt{void input ([подходящ тип] array, int n)}, която въвежда от клавиатурата стойностите на елементите на масива \texttt{array} от произволен тип \texttt{T} с големина \texttt{n}. \\

	\textit{Какви са допустимите типове \texttt{T} за този шаблон? Защо функцията е от тип \texttt{void}?}\\

	Да се реализира и изпълни подходящ тест за функцията.

	\item Да се реализира шаблон на фукнция \texttt{bool ordered ([подходящ тип] array, int n)}, която проверява дали елементите на масива \texttt{array} от произволен тип \texttt{T} с големина \texttt{n} образуват монотонно-растяща редица спрямо релацията \texttt{<}.\\

	\textit{Какви са допустимите типове \texttt{T} за този шаблон?}\\

	Да се реализира и изпълни подходящ тест за функцията.

	\item Да се реализира шаблон на фукнция \texttt{bool member ([подходящ тип] array, int n, [подходящ тип]x)}, която проверява дали \texttt{x} е елемент на масива \texttt{array} от произволен тип \texttt{T} с големина \texttt{n}.\\

	\textit{Има ли в C++ тип \texttt{T}, който не е съвместим с този шаблон?}\\

	Да се реализира и изпълни подходящ тест за функцията.

	\item Да се дефинира масив \texttt{functions} с 5 елемента от тип фунцкия $double \rightarrow double$. Да се дефинират 5 произволни функции от този тип и адресите им да се се присвоят на елементите на масива.\\

	При въведено от клавиатурата число $x:double$, да се намери и отпечата индексът на тази функция в масива \texttt{functions}, чиято стойност е най-голяма в точката \texttt{x} спрямо стойностите на всички функции в масива. Ако има няколко такива функции, да се отпечата индекса на коя да е от тях. 

	\item Да се дефинира функция \texttt{double fmax([подходящ тип]f, [подходящ тип]g, double x)}, където \texttt{f} и \texttt{g} са две произволни функции от тип $double \rightarrow double$, за които приемаме, че са дефинирани в \texttt{x}.  Функцията да връща по-голямата измежду стойностите на \texttt{f} и \texttt{g} в точката x.\\

	Да се реализира и изпълни подходящ тест за функцията.

	\item Да се дефинира фунцкия \texttt{double maxarray ([подходящ тип] array, int n, double x)}, където \texttt{array} е масив от финкции от тип $double \rightarrow double$ с големина \texttt{n}. \\

	Фунцкцията \texttt{maxarray} да връща най-голямата измежду стойностите на всички функции от масива в точката \texttt{x} като приемаме, че всички те са дефинирани в тази точка.\\

	Задачата да се реши със и без използването на функцията \texttt{fmax} от предишната задача.\\

	Да се реализира и изпълни подходящ тест за функцията.

	\item Да се дефинира функция \texttt{void map (double array[], int n, [подходящ тип]f)}, където \texttt{array} е масив от стойности от тип $double$ с големина \texttt{n}, a f е функция от тип $double \rightarrow double$.\\

	Функцията да заменя всяка стойност \texttt{array[i]} на масива \texttt{array} със стойността \texttt{f(array[i])}.\\

	Например, ако е даден масивът \texttt{double a[] = \{1,2,3\}} и функцията \texttt {double inc (double x) \{return x+1;\}}, то след изпълнението на \texttt{map (a,3,inc)}, елементите на масива \texttt{a} ще имат стойности съответно 2, 3 и 4.\\

	Да се реализира и изпълни подходящ тест за функцията.

	\item Нека е дадена следната структура: \texttt{struct S \{int a; int b; int c;\};}. Да се дефинира функция \texttt{void sort([подходящ тип]array, int n, [подходящ тип]compare)}, където \texttt{array} е масив от \texttt{n} структури от тип \texttt{S}.\\

	Типът на функцията \texttt{compare} да се подбере така, че чрез нея да може да се реализира произволна наредба за типа \texttt{S}, т.е. функцията да може да сравнява ``по големина'' две структури от \texttt{S} по произволен критерий.\\

	Да се създаде и инициализира масив с 5 структури от тип \texttt{S}. Като се използва функцията \texttt{sort} да се сортира масива по веднъж по всеки от следните начини:

	\begin{enumerate}
		\item по полето \texttt{a}
		\item по полето \texttt{b}
		\item лексикографски по тройката $(a,b,c)$
	\end{enumerate}

\end{enumerate}




\end{document}

