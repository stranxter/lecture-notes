\documentclass[12pt,a4paper]{article}
\usepackage[utf8]{inputenc}
\usepackage{amsmath}
\usepackage{amsfonts}
\usepackage{amssymb}
\usepackage{listings}
\usepackage{url}
\usepackage[bulgarian]{babel}
\usepackage{listings}
\usepackage{enumerate}
\usepackage{hyperref}
\usepackage[framemethod=tikz]{mdframed}
\usepackage{relsize}


\newcommand{\code}[1]{\texttt{#1}}

\lstset{breaklines=true} 


\author{\textit{email: kalin@fmi.uni-sofia.bg}}
\title{\textsc{Задачи за задължителна самоподготовка} \\
по \\
Обектно-ориентирано програмиране\\
\textit{Наследяване и сериализация}}



\begin{document}
\maketitle


\begin{enumerate}

	
	\item Да се сериализира и десериализира масив от масиви от числа:

	 \code{DynArray<DynArray<int> >}

	Да се тества програмата!

	\item Да се сериализира и десериализира масив от масиви от символни низове:

	 \code{DynArray<DynArray<char*> >}

	Да се тества програмата!

	\item Към задачата за Шахматните фигури да се реализира сериализация и десериализация на масив от шахматни фигури.

	Да се тества програмата!

	\item Към задачата \code{2.4.25.} за софтуерна фирма да се реализира сериализация и десериализация на масив от служители.

	Да се тества програмата!

	\item Да се реализира абстрактен клас \code{NetworkDevice}, който дефинира следните (абстрактни, чисто втируални) операции:

	\begin{itemize}
		\item \code{bool attachTo(NetworkDevice* device)}: свързва устройството с друго устройство \code{device}. 

		Реализаициите на метода в наследниците на \code{NetworkDevice} ще връщат \code{true}, ако свързването е възможно и \code{false}, ако свързването не е възможно.  Правилата, по които се определя дали може или не може да се свърже устройството, зависят от конкретния наследник на \code{NetwrokDevice}. 

		При дефиниране на метода в наследените класове осигурете, че създадената връзка е двупосочна. Т.е. счита се, че ако устройството \code{A} е свързано с устройството \code{B}, то и устройството \code{B} е свързано с устройството \code{A}. Не допускайте дадено устройство да може да се свърже повече от веднъж с едно и също устройство или пък да се свърже със себе си.

		\item \code{[попълнете правилния тип] getAttachedDevice(int i)}. Връща (\textit{Упътване: указател към}) i-тото поред устройство, към което даденото устройство е свързано и NULL, ако индексът i не е валиден.

	\end{itemize}

	Класът \code{NetworkDevice} също да съдържа и уникален идентификатор от тип \code{int} за всяко устройство. Идентификаторът може да се задава чрез конструкторите на наследниците.

	Да се реализират производните класове \code{EndDevice} и \code{Switch}. 

	\begin{itemize}
		\item За \code{EndDevice} е характерно, че устройството може да е свързано най-много с още едно устройство. Т.е. във всеки момент от времето \code{EndDevice} или не е свързан с нищо, или е свързан с точно едно устройство. Веднъж свързано, \code{EndDevice} не може да бъде свързано отново с друго устройство.

		\item За Switch е характерно, че устройството може да е свързано с максимум 8 други устройства. При достигане на броя на свързаните устройства до 8, устройството \code{Swith} не може да бъде свързвано с повече устройства.

	\end{itemize}

	Промяна на вече създадена връзка не е възможна и при двата вида устройства.

	За така дефинираните класове да се решат следните задачи:


	\begin{enumerate}
		\item Да се реализира функкция \code{void printConnections(NetworkDevice* devices[],int n)}, която отпечатва на екрана информация за връзките на всяко устройство в масив от устройства. \textit{Упътване: добавете нова виртуална функция printConncection в базовия клас. Връзките можете да печатате във формата \code{<идентификатор 1> --- <идентификатор 2>}}.

		\item Да се създаде примерна програма, в която се инициализират няколко устройства, добавят се в масив, създават се връзки между тях и се използва функцията \code{printConnections} за отпечатване на връзките.

		\item Да се реализира функция \code{bool connected([попълнете правилния тип] d1, [попълнете правилния тип] d2)}, която проверява дали има връзка (пряка или косвена) между устройствата d1 и d2. За улеснение приемете, че няма циклични връзки между устройствата. \textit{Упътване: използвайте рекурсивна функция по подобие на функциите за търсене на път.}

		\item \textit{Допълнителна задача:}Решете горната задача и при случая на възможни циклични връзки.

		\item \textit{Допълнителна задача:}Напишете функция, която по масив от устройства търси дали има циклична връзка между някои две от тях.

		\item \textit{Допълнителна задача с повишена трудност:} Сериализирайте и десериализирайте масив от свързани устройства.
	\end{enumerate}

	\item Да се сериализира и десериализира масив от Скутери и Батмобили. 
	Да се тества програмата!

	


\end{enumerate}







	\vspace{20px}


\end{document}

