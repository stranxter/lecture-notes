\documentclass[12pt,a4paper]{article}
\usepackage[utf8]{inputenc}
\usepackage{amsmath}
\usepackage{amsfonts}
\usepackage{amssymb}
\usepackage{listings}
\usepackage{url}
\usepackage[bulgarian]{babel}
\usepackage{listings}
\usepackage{enumerate}


\newcommand{\code}[1]{\texttt{#1}}

\lstset{breaklines=true}


\author{\textit{email: kalin@fmi.uni-sofia.bg}}
\title{\textsc{Задачи за задължителна самоподготовка} \\
по \\
Обектно-ориентирано програмиране\\
\textit{Копиране}}



\begin{document}
\maketitle


\begin{enumerate}

\item За клас \code{BrowserHistory} от предишните домашни да се реализират конструктор за копиране, оператор за присвояване, оператори за събиране \code{+} и \code{+=}, обединяващи две истории и деструктор.\\

Да се реализира подходящ тест на класа.

\item Клас \code{Dictionary} от предишните домашни да се реализира така, че максималният брой \code{N} на двойки ключ-стойност, които могат да бъдат добавени към речника, да се задава като параметър на конструктора на класа. За класа да се реализират конструктор за копиране, оператор за присвояване и деструктор.\\

Да се реализират оператори за събиране \code{+} и \code{+=}, обединяващи два речника. Ако в речниците \code{a} и \code{b} има еднакви думи с различни значения, то за тези думи в речника \code{a+b} да се използва значението им от речника \code{a}. \\

Да се реализира подходящ тест на класа.

\item За клас \code{Dynarray} от лекции да се дефинира метод \code{Dyanarray::resize}, с който да може динамично да се променя капацитета на масива. При намаляване на капацитета да отпадат най-левите елементи на масива. При увеличаване на капацитета на масива, новите елементи да остават неинициализрани.

Да се реализират подходящи тестове.

\item За клас \code{Dynarray} от лекции да се дефинира метод \code{Dyanarray::slice(size\_t n)}. Ако приемем, че изходният масив е  с елементи от тип T, то методът \code{slice} трябва да връща масив, състоящ се от масиви с елементи от тип T. \code{i}-тият поред масив от резултата съдържа i-тата n-торка от последователни членове на изходния масив. Последният масив в резултата може да съсъдржа по-малко от \code{n} елемента, ако броят на елементите на изходният масив не е кратен на \code{n}.\\

Пример: Нека масивът \code{a} има елементите \code{[1,2,3,4,5,6,7,8,10,11]}. При тези условия, \code{a::slice(3)} създава и връща масива от масиви \code{[[1,2,3],[4,5,6],[7,8,9],[10,11]]}.




\end{enumerate}


	\vspace{20px}


\end{document}
