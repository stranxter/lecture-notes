\documentclass[12pt,a4paper]{article}
\usepackage[utf8]{inputenc}
\usepackage{amsmath}
\usepackage{amsfonts}
\usepackage{amssymb}
\usepackage{listings}
\usepackage{url}
\usepackage[bulgarian]{babel}
\usepackage{listings}
\usepackage{enumerate}
\usepackage{hyperref}
\usepackage[framemethod=tikz]{mdframed}
\usepackage{relsize}


\newcommand{\code}[1]{\texttt{#1}}

\lstset{breaklines=true}


\author{\textit{email: kalin@fmi.uni-sofia.bg}}
\title{\textsc{Задачи за задължителна самоподготовка} \\
по \\
Обектно-ориентирано програмиране\\
\textit{Линеен едносвързан списък}}



\begin{document}
\maketitle

Следните задачи да се решат като упражнение за директно боравене с указателите и двойните кутии, вместо да се свеждат до използването на вече готови методи от реализацията на класа. Т.е. решенията на задачите да не ползват други методи, освен ако не са помощни функции, специално написани за тях.

\begin{enumerate}

	\item Да се реализират следните операции в шаблона LList, разработен на лекции:

	\begin{enumerate}
		\item метод \code{get\_ith(int n)} за намиране на \code{n}-тия поред елемент на списъка
		\item метод \code{push\_back} за дабавяне на елемент от тип \code{T} към \textit{края} на списъка
		\item Метод \code{removeAll (x)}, който изтрива всички срещания на елемента \code{x} от списъка.
		\item Метод \code{$l_1$.append($l_2$)}, която добавя към края на списъка $l_1$ всички елементи на списъка $l_2$.
		\item Метод \code{concat}, който съединява два списъка в нов, трети списък. Т.е. \code{$l_1$.concat($l_2$)} създава и връща нов списък от елементите на \code{$l_1$}, следвани от елементите на \code{$l_2$}.
		\item Да се дефинират оператори \code{+=} и \code{+}, съответни на методите \code{append} и \code{concat}.
		\item оператор за индексиране, позволяващ чете и писане
		\item Метод \code{reverse}, който обръща реда на елементите на списъка. Например, списъкът с елементи $1,2,3$ ще се преобразува до списъка с елементи $3,2,1$.
	\end{enumerate}
\end{enumerate}


	\vspace{20px}


\end{document}
