\documentclass[12pt,a4paper]{article}
\usepackage[utf8]{inputenc}
\usepackage{amsmath}
\usepackage{amsfonts}
\usepackage{amssymb}
\usepackage{listings}
\usepackage{url}
\usepackage[bulgarian]{babel}
\usepackage{listings}
\usepackage{enumerate}


\lstset{breaklines=true} 


\author{\textit{email: kalin@fmi.uni-sofia.bg}}
\title{\textsc{Задачи за задължителна самоподготовка} \\
по \\
Увод в програмирането\\
\textit{Указатели, масиви, аргументи към функции}}



\begin{document}
\maketitle


\begin{enumerate}
	
	\item Задача 3.55. 

	Дадена е квадратна таблица $A_{n\times n}$ $(1 \le n \le 30)$ от низове, съдържащи думи с максимална дължина 6. Да се напише програма, която проверява дали изречението, получено след конкатенацията на думите от главния диагонал (започващо от горния ляв ъгъл) съвпада с изречението, получено след конкатенацията на думите от вторичния главен диагонал на A (започващо от долния ляв ъгъл).

	\item Задача 3.56. 

	Дадена е квадратна таблица A от n-ти ред $(1 \le n \le 20)$ от низове, съдържащи думи с максимална дължина 9. Да се напише програма, която намира и извежда на екрана изречението, получено след обхождане на A по спирала в посока на движението на часовниковата стрелка, започвайки от горния ляв ъгъл. Например ако матрицата A има вида:

$\left( \begin{array}{ccc}
a & b & c \\
d & e & f \\
g & h & i 
\end{array} \right)$

	изречението след обхождането по спирала е: ``abcfihgde''.
		
	\item Задача 3.57 (Inner Join). 

	Нека са дадени два масива от низове – students и grades с най-много 20 низа във всеки. Низовете в масива students имат вида ``XXXXXX YYYY...'', където ``XXXXXX'' е шестцифрен факултетен номер, а ``YYYY...'' е име с произволна дължина. Низовете в grades имат вида ``XXXXXX YYYY'', където ``XXXXXX'' е шестцифрен факултетен номер, а ``YYYY'' е оценка под формата на число с плаваща запетая. И двата масива са сортирани във възходящ ред по факултетен номер. Възможно е в някой от двата масива да има данни за факултетни номера, за които няма данни в другия. И в двата списъка даден факултетен номер се среща най-много един път. Да се напише програма, която извежда на екрана имената и оценките на тези студенти, за които има информация и в двата списъка, като оценките са увеличени с 1 единица, но са максимум 6.00.


	\item Да се дефинира функция

	\texttt{swap([подходящ тип] a,[подходящ тип] b)},

	която разменя стойностите на целочислените променливи a и b. Задачата да се реши по два начина - чрез използване на указател и на псевдоним.


	\item Да се дефинира функция, която получава като параметри два масива с еднакъв брой елементи. Функцията да разменя съответните елементи на масивите ($a[i] \leftrightarrow b[i]$).

	\item Да се напише булева функция, която получава като параметър масив от указатели към целочислени променливи. Функцията да проверява дали поне две от съответните променливи имат еднакви стойности.

	\item Да се дефинира функцията 

	\texttt{bool subarrays (int *arrays[],int npointers, int arrlengths[])}.

	Масивът \texttt{arrays} съдъръжа \texttt{npointers} на брой указатели към масиви от цели числа. \texttt{i}-тият масив има големина \texttt{arrlengths[i]}. Функцията да връща истина, ако поне един от масивите е подмасив на друг масив. Масивът $a$ наричаме подмасив на $b$, ако заетата от $a$ памет е част от заетата от $b$ памет.


		\item Да се дефинира функцията 

	\texttt{bool commonel (int *arrays[],int npointers, int arrlengths[])}.

	Масивът \texttt{arrays} съдъръжа \texttt{npointers} на брой указатели към масиви от цели числа. \texttt{i}-тият масив има големина \texttt{arrlengths[i]}. Функцията да връща истина, ако има поне едно число $x$, което е елемент на всички масиви.


\end{enumerate}


	\vspace{20px}

	\small{Някои от задачите са от сборника \textit{Магдалина Тодорова, Петър Армянов, Дафина Петкова, Калин Николов, ``Сборник от задачи по програмиране на C++. Първа част. Увод в програмирането''}. За тези задачи е запазена номерацията в сборника.}



\end{document}

