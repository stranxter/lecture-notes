\documentclass[12pt,a4paper]{article}
\usepackage[utf8]{inputenc}
\usepackage{amsmath}
\usepackage{amsfonts}
\usepackage{amssymb}
\usepackage{listings}
\usepackage{url}
\usepackage[bulgarian]{babel}
\usepackage{listings}
\usepackage{enumerate}


\lstset{breaklines=true} 


\author{\textit{email: kalin@fmi.uni-sofia.bg}}
\title{\textsc{Задачи за задължителна самоподготовка} \\
по \\
Увод в програмирането\\
\textit{базови програми}}



\begin{document}
\maketitle


\begin{enumerate}
	\item Задача 1.6.

	Да се напише програма, която по зададени навършени години намира приблизително броя на дните, часовете, минутите и секундите, които е живял човек до навършване на зададените години.

	\item Задача 1.7.

	 Да се напише програма, която намира лицето на триъгълник по дадени: а) дължини на страна и височина към нея; б) три страни.

	 \item Задача 1.14.

	 Да се запише булев израз, който да има стойност истина, ако посоченото условие е вярно и стойност - лъжа, в противен случай:


	 \begin{enumerate}[a)] % a), b), c), ...
		\item цялото число p се дели на 4 или на 7;
		\item уравнението $ax^2 + bx + c = 0 (a \neq 0)$ няма реални корени;
		\item точка с координати (a, b) лежи във вътрешността на кръг с радиус 5 и център (0, 1); г) точка с координати (a, b) лежи извън кръга с център (c, d) и радиус f;
		\item точка принадлежи на частта от кръга с център (0, 0) и радиус 5 в трети квадрант;
		\item точка принадлежи на венеца с център (0, 0) и радиуси 5 и 10;
		\item x принадлежи на отсечката [0, 1];
		\item x е равно на max \{a, b, c\};
		\item x е различно от max \{ a, b, c\};
		\item поне една от булевите променливи x и y има стойност true;
		\item и двете булеви променливи x и y имат стойност true;
		\item нито едно от числата a, b и c не е положително;
		\item цифрата 7 влиза в записа на положителното трицифрено число p;
		\item цифрите на трицифреното число m са различни;
		\item поне две от цифрите на трицифреното число m са равни помежду си;
		\item цифрите на трицифреното естествено число x образуват строго растяща или строго намаляваща редица;
		\item десетичните записи на трицифрените естествени числа x и y са симетрични;
		\item естественото число x, за което се знае, че е по-малко от 23, е просто.

	\end{enumerate}

	\item Задача 1.20.

	Да се напише програма, която по въведени от клавиатурата цели числа x и k ($k \geq 1$) намира и извежда на екрана k-тата цифра на х. Броенето да е отдясно наляво.

	\item Задача 2.7.

	Да се напише програма, която въвежда координатите на точка от равнина и извежда на кой квадрант принадлежи тя. Да се разгледат случаите, когато точката принадлежи на някоя от координатните оси или съвпада с центъра на координатната система.

	\item Задача 2.12.

	Да се напише програма, която проверява дали дадена година е високосна.

	\item Задача 2.40.

	Да се напише програма, която (чрез цикъл for) намира сумата на всяко трето цяло число, започвайки от 2 и ненадминавайки n (т.е. сумата 2 + 5 + 8 + 11 + ...).

	\item Задача 2.44.

	Дадено е естествено число n ($n \geq 1$). Да се напише програма, която намира броя на тези елементи от серията числа $i^3 + 13 \times i \times n + n
	^3$ , $i = 1, 2, ..., n$, които са кратни на 5 или на 9.

	\vspace{20px}

	\small{Задачите са от сборника \textit{Магдалина Тодорова, Петър Армянов, Дафина Петкова, Калин Николов, ``Сборник от задачи по програмиране на C++. Първа част. Увод в програмирането''}. Запазена е номерацията в сборника.}

\end{enumerate}




\end{document}

