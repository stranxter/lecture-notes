\documentclass{beamer}
\usepackage{relsize}
\usepackage{color}

\usepackage{listings}
\usetheme{CambridgeUS}
%\usepackage{beamerthemesplit} % new 
\usepackage{enumitem}
\usepackage{amsmath}                    % See geometry.pdf to learn the layout options. 
\usepackage{amsthm}                   % See geometry.pdf to learn the layout options. There 
\usepackage{amssymb}                    % See geometry.pdf to learn the layout options. 
\usepackage[utf8]{inputenc} 
\usepackage{graphicx}
\usepackage[english,bulgarian]{babel}
\usepackage[framemethod=tikz]{mdframed}
\usepackage{caption}
\usepackage{tikz}
\usepackage{forest}
\usetikzlibrary{shapes,arrows,positioning,calc,chains}

\lstset{language=C++,
                basicstyle=\ttfamily,
                keywordstyle=\color{blue}\ttfamily,
                stringstyle=\color{red}\ttfamily,
                commentstyle=\color{green}\ttfamily,
                morecomment=[l][\color{magenta}]{\#}
}

\newtheorem{mydef}{Дефиниция}[section]
\newtheorem{lem}{Лема}[section]
\newtheorem{thm}{Твърдение}[section]

\DeclareMathOperator{\restrict}{\upharpoonright}

\setitemize{label=\usebeamerfont*{itemize item}%
  \usebeamercolor[fg]{itemize item}
  \usebeamertemplate{itemize item}}

\setbeamercovered{transparent}

\captionsetup{font=footnotesize}

\lstset{breaklines=true}
\tikzset{
block/.style = {draw, fill=white, rectangle,align = center},
entry/.style = {draw, fill=black, circle, radius=3em},
condition/.style = {draw, fill=white, diamond, align = center,node distance=3cm},
fork/.style = {draw, fill=black, circle,inner sep=1pt},
lnode/.style={rectangle split, rectangle split parts=3,draw, rectangle split horizontal},
treenode/.style = {align=center, inner sep=0pt, text centered, circle, font=\sffamily\bfseries, draw=black, fill=white, text width=1.5em}
}


\begin{document}
\title[Обектно-ориентирано програмиране]{Move семантика} 
\author{Калин Георгиев} 
\frame{\titlepage} 


\section{Преместване вместо копиране} 

\begin{frame}[fragile]
  \frametitle{Кога копирането е проблем?}
  \begin{itemize}
    \item \verb#String s(makeSomeString());#
    \item \verb#s=makeSomeString();#
  \end{itemize}

  \bigskip
  И това, но то се оптимизира:
  \bigskip

  \begin{itemize}
    \item \verb#return String("hello");#
  \end{itemize}
\end{frame}


\begin{frame}[fragile]
  \frametitle{Lvalues и Rvalues}

  \begin{itemize}
    \item Кои са възможните lvalues: \texttt{x}, \texttt{a[i]}, \texttt{*p}, \texttt{f(x)}, ...
    \item Нещо, на което можеш да му взешем адреса. Локатор
    \item Кои са възможните rvalues: \texttt{42}, \texttt{f(x)}
    \item Нещо, което е временен обект
  \end{itemize}

  \bigskip

  Как да ``засечем'' rvalue? Rvalue референции: \verb#String&& s#

    
\end{frame}


\begin{frame}[fragile]
  \frametitle{Как се прави \texttt{move} конструктор и оператор =?}

\begin{lstlisting}[basicstyle=\small]
  String::String(String&& other)
  {
    str = other.str;
    other.str = nullptr;
  }

  String& operator=(String&& other)
  {
    delete str;
    str = other.str;
    other.str = nullptr;
  }
\end{lstlisting}  

\end{frame}

\begin{frame}[fragile]
  \frametitle{\texttt{std::move}: Да форсираме преместване}

\begin{lstlisting}[basicstyle=\small]
  class Person
  {
    String name;
    //...
    Person::Person(Person&& other)
                    :name(other.name){}
  };
\end{lstlisting}  

Vs.

\begin{lstlisting}[basicstyle=\small]
    Person::Person(Person&& other)
                    :name(std::move(other.name)){}
\end{lstlisting}  

\end{frame}


\begin{frame}
\centerline{Благодаря за вниманието!}
\end{frame}


\end{document}

\begin{columns}[t]
  \begin{column}{0.55\textwidth}

  \end{column}
  \begin{column}{0.45\textwidth}

  \end{column}
\end{columns}


