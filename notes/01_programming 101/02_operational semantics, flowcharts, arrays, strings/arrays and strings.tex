\documentclass{beamer}
\usepackage{relsize}
\usepackage{color}

\usepackage{listings}
\usetheme{CambridgeUS}
%\usepackage{beamerthemesplit} % new 
\usepackage{enumitem}
\usepackage{amsmath}                    % See geometry.pdf to learn the layout options. 
\usepackage{amsthm}                   % See geometry.pdf to learn the layout options. There 
\usepackage{amssymb}                    % See geometry.pdf to learn the layout options. 
\usepackage[utf8]{inputenc} 
\usepackage{graphicx}
\usepackage[english,bulgarian]{babel}

\lstset{language=C++,
                basicstyle=\ttfamily,
                keywordstyle=\color{blue}\ttfamily,
                stringstyle=\color{red}\ttfamily,
                commentstyle=\color{green}\ttfamily,
                morecomment=[l][\color{magenta}]{\#}
}

\newtheorem{mydef}{Дефиниция}[section]
\newtheorem{lem}{Лема}[section]
\newtheorem{thm}{Твърдение}[section]

\DeclareMathOperator{\restrict}{\upharpoonright}

\setitemize{label=\usebeamerfont*{itemize item}%
  \usebeamercolor[fg]{itemize item}
  \usebeamertemplate{itemize item}}

\setbeamercovered{transparent}



\begin{document}
\title[Увод в програмирането]{Масиви и символни низове} 
\author{Калин Георгиев} 
\frame{\titlepage} 


\section{Масиви и низове} 

\begin{frame}
\centerline{Първоначални сведения за масиви и низове}
\end{frame}

\begin{frame}[fragile]
\frametitle{Представяне на низове}

\begin{itemize}
  \item ASCii таблица 
\end{itemize}

\begin{tabular}{c | c | c | c | c}
\hline
...&$A^{65}$&$B^{66}$&$C^{67}$&... \\\hline
  
\end{tabular}

\pause

\begin{itemize}
  \item Представяне в паметта 
\end{itemize}

\begin{tabular}{c | c | c | c | c | c | c | c}
\hline
...&H &E &L &L &O & \_ &... \\\hline
...&72&69&76&76&79& 0 &... \\\hline
\end{tabular}

\pause
\begin{itemize}
  \item Масиви
\end{itemize}


\end{frame}

\begin{frame}[fragile]
\frametitle{Масиви}

\begin{itemize}
  \item Дефиниране чрез тип и размер: 

\begin{lstlisting}
int arr[100];
\end{lstlisting}
\pause
  \item Достъп до всеки отделен елемент:

\begin{lstlisting}
int b = arr[18];
cin >> arr[18];
b = arr[1] + arr[2];
\end{lstlisting}
\pause

  \item Обхождане с for цикъл

\begin{lstlisting}
for (int count = 0; count < 100; count++)
{
  cout << arr[count];
}
\end{lstlisting}
\end{itemize}

\end{frame}


\begin{frame}[fragile]
\frametitle{Прост пример с низове}

\begin{flushleft}
\relscale{0.5}
\begin{lstlisting}
int main ()
{

  char str[100] = "Hello world!";
  cout << str << endl;

  str[0] = 'Y';
  cout << str << endl;

  cout << "Please input a string:";
  cin >> str;

  for (int counter = 0; counter < 100; counter++)
  {
    if (str[counter] == 'a')
    {
      str[counter] = 'b';
    }
  }

  cout << str << endl;

  return 0;
}
\end{lstlisting}
\end{flushleft}

\end{frame}


\begin{frame}[fragile]
\frametitle{Какво не можем да правим с масиви и низове}

\begin{itemize}
  \item Няма проверка за коректност!
\end{itemize}

\begin{flushleft}
\relscale{0.75}
\begin{lstlisting}
char a[6] = "HELLO";
a[10] = '!';
\end{lstlisting}


\begin{tabular}{c | c | c | c | c | c | c | c | c | c}
\hline
...&0 &1 &2 &3 &4 &5   &...&10&... \\\hline
...&H &E &L &L &O & \_ &...&\alert{!}& ... \\\hline
\end{tabular}
\end{flushleft}

\pause

\begin{itemize}
  \item Присвояване (a=b)
  \pause
  \item Сравнение (a==b, a < b,...)
\end{itemize}

\end{frame}


\begin{frame}[fragile]
\frametitle{Вградени функции за работа с низове}

\begin{columns}[t]

  \begin{column}{0.6\textwidth}
\relscale{0.5}
  \textbf{Дължина на низ}
\begin{lstlisting}
cout << strlen(a);
\end{lstlisting}
  \textbf{Присвояване на низове}
\begin{lstlisting}
strcpy (d,a);
strcpy (c,a); //!!!
\end{lstlisting}
  \textbf{Сравнение на низове}
\begin{lstlisting}
if (strcmp (a,b) < 0) 
  {cout << "a<b";} 
else 
  {cout << "b<=a";};
\end{lstlisting}
  \textbf{Конкатенация на низове}
\begin{lstlisting}
strcpy (d,a); //d -> "HELLO"
\end{lstlisting}


\begin{tabular}{c | c | c | c | c | c | c | c | c | c | c | c | c }
\hline
...&0 &1 &2 &3 &4 &5 &6 &7 &8 &9 &10   &... \\\hline
...&H &E &L &L &O &\_ &? &? &? &? &? &... \\\hline
\end{tabular}

\begin{lstlisting}
strcat (d,b); //d -> "HELLOWORLD"
\end{lstlisting}

\begin{tabular}{c | c | c | c | c | c | c | c | c | c | c | c | c }
\hline
...&0 &1 &2 &3 &4 &5 &6 &7 &8 &9 &10   &... \\\hline
...&H &E &L &L &O &W &O &R &L &D &\_ &... \\\hline
\end{tabular}



  \end{column}


  \begin{column}{0.4\textwidth}

\begin{flushleft}
\relscale{0.5}
\begin{lstlisting}
#include <cstring>
...
char a[6] = "HELLO";
char b[6] = "WORLD";
char c[4] = "BYE";
char d[11] = "???";
\end{lstlisting}

\begin{tabular}{c | c | c | c | c | c | c | c }
\hline
...&0 &1 &2 &3 &4 &5   &... \\\hline
...&H &E &L &L &O & \_ &... \\\hline
\end{tabular}

\begin{tabular}{c | c | c | c | c | c | c | c }
\hline
...&0 &1 &2 &3 &4 &5   &... \\\hline
...&W  &O &R &L &D & \_ &... \\\hline
\end{tabular}

\begin{tabular}{c | c | c | c | c | c  }
\hline
...&0 &1 &2 &3   &... \\\hline
...&B &Y &E & \_ &... \\\hline
\end{tabular}

\end{flushleft}

  \end{column}
\end{columns}


\end{frame}

\end{document}
