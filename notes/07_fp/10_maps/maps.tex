\documentclass{beamer}
\usepackage{relsize}
\usepackage{color}

\usepackage{listings}
\usetheme{CambridgeUS}
%\usepackage{beamerthemesplit} % new
\usepackage{enumitem}
\usepackage{amsmath}                    % See geometry.pdf to learn the layout options.
\usepackage{amsthm}                   % See geometry.pdf to learn the layout options. There
\usepackage{amssymb}                    % See geometry.pdf to learn the layout options.
\usepackage[utf8]{inputenc}
\usepackage{graphicx}
\usepackage[english,bulgarian]{babel}

\usepackage{caption}
\usepackage{tikz}
\usepackage{forest}

\usetikzlibrary{shapes,arrows,positioning,calc,positioning,fit,chains}

\usetheme{CambridgeUS}
\usecolortheme{crane}

\lstset{language=C++,
                basicstyle=\ttfamily,
                keywordstyle=\color{blue}\ttfamily,
                stringstyle=\color{red}\ttfamily,
                commentstyle=\color{green}\ttfamily,
                morecomment=[l][\color{magenta}]{\#}
}

\tikzset{
block/.style = {draw, fill=white, rectangle,align = center},
entry/.style = {draw, fill=black, circle, radius=3em},
condition/.style = {draw, fill=white, diamond, align = center,node distance=3cm},
fork/.style = {draw, fill=black, circle,inner sep=1pt},
lnode/.style={rectangle split, rectangle split parts=3,draw, rectangle split horizontal},
treenode/.style = {align=center, inner sep=0pt, text centered, circle, font=\sffamily\bfseries, draw=black, fill=white, text width=1.5em},
flexnode/.style = {align=center, text centered, ellipse, font=\sffamily\bfseries, draw=black, fill=white},
token/.style={rectangle split, rectangle split parts=2,draw, rectangle split horizontal=false}
}


\newtheorem{mydef}{Дефиниция}[section]
\newtheorem{lem}{Лема}[section]
\newtheorem{thm}{Твърдение}[section]

\DeclareMathOperator{\restrict}{\upharpoonright}

\setitemize{label=\usebeamerfont*{itemize item}%
  \usebeamercolor[fg]{itemize item}
  \usebeamertemplate{itemize item}}

\setbeamercovered{transparent}

\captionsetup{font=tiny} 

\begin{document}
\title[Речници]{Префиксни дървета. Речници}
\frame{\titlepage}

\section{Какво е речник (``map'', ``dictionary'')}
\subsection{}


\begin{frame}[fragile]
\frametitle{Речник}
\begin{center}
\begin{tabular} {c | l}
  cat & a small animal that is related \\
      & to lions and tigers and that is \\
      & often kept by people as a pet \\
      \hline\\
  dog & canid; especially :  a highly variable \\
      & domestic mammal (Canis familiaris) closely \\
      & related to the gray wolf
\end{tabular}  
\end{center}
\end{frame}


\begin{frame}[fragile]
  \frametitle{Други изображения}

Информация, най-често обемна и структурирана, която се асоциира с ключ: проста, атомарна, уникална стойност, която идентифицира информацията.

\bigskip

\begin{itemize}
  \item Телефонен указател
  \item Хистограми
  \item Кешове
  \item Конфигурации
  \item Сесийни ключове
\end{itemize}

\end{frame}  



\begin{frame}[fragile]
\frametitle{Променливи на средата}

\begin{columns}[t]
  \begin{column}{0.5\textwidth}
\begin{flushleft}
\relscale {0.7}
\begin{lstlisting}
 var x = 10;
 var y = "Hello"; 
\end{lstlisting}
\end{flushleft}
  \end{column}

  \begin{column}{0.5\textwidth}
\begin{flushleft}
\begin{tabular} {c | l}
  x   & 10 \\
      \hline\\
  y & ``hello''

\end{tabular}  
\end{flushleft}

  \end{column}
\end{columns}


\begin{columns}[t]
  \begin{column}{0.5\textwidth}
\begin{flushleft}
\relscale {0.7}
\begin{lstlisting}
 var x = 10;
 var y = "Good bye!"; 
\end{lstlisting}
\end{flushleft}
  \end{column}

  \begin{column}{0.5\textwidth}
\begin{flushleft}
\begin{tabular} {c | l}
  x   & 10 \\
      \hline\\
  y & ``Good bye!''

\end{tabular}  
\end{flushleft}

  \end{column}
\end{columns}
\end{frame}

\begin{frame}
  \centerline{Асоциативни списъци}
\end{frame}

\begin{frame}[fragile]
  \frametitle{Асоциативни списъци}

\begin{lstlisting}[basicstyle=\small,language=Haskell]
type Assoc k v = [(k,v)]

ghci> :t lookup
lookup :: Eq a => a -> [(a, b)] -> Maybe b
\end{lstlisting}

\end{frame}

\begin{frame}
  \centerline{Префиксни дървета (Trie)}
\end{frame}

\begin{frame}[fragile]
  \frametitle{Trie}
  

\begin{columns}[t]
  \begin{column}{0.20\textwidth}


  	\begin{tabular}{c | c}
  		\textit{key} & \textit {value} \\\hline
  		to & 7 \\
  		tea & 3 \\
  		ted & 4 \\
  		ten & 12 \\
  		A & 15 \\
  		i & 11 \\
  		in & 5 \\
  		inn & 9 \\
  	\end{tabular}

  \end{column}
  \begin{column}{0.80\textwidth}
    \relscale{0.8}
    \begin{figure}
      \centering
      \begin{forest}
      for tree={align=center, inner sep=0pt, text centered, circle, font=\sffamily\bfseries, draw=black, fill=white, text width=1.5em }
      [\_, name=ROOT
        [\_, name=ROOT_T, edge label={node[midway,left] {\small{t}}}
          [7, name=ROOT_TO, edge label={node[midway,left] {\small{o}}}]
          [\_, name=ROOT_TE, edge label={node[midway,left] {\small{e}}}
            [3, edge label={node[midway,left] {\small{a}}}]
            [4, edge label={node[midway,left] {\small{d}}}]
            [12, edge label={node[midway,left] {\small{n}}}]
          ]
        ]
        [15, name=ROOT_A, edge label={node[midway,right] {\small{A}}}]
        [11, name= ROOT_i, edge label={node[midway,right] {\small{i}}}
          [5, edge label={node[midway,left] {\small{n}}}
            [9, edge label={node[midway,left] {\small{n}}}]
          ]
        ]
      ]
      %edge label={node[midway,left] {\small{Help!}}}
      \end{forest}
      \label{fig:trie1}
      \end{figure}
    
  \end{column}
\end{columns}

\end{frame}


\begin{frame}[fragile]
  \frametitle{Добавяне на ключ в Trie}
  

\begin{columns}[t]
  \begin{column}{0.20\textwidth}


  	\begin{tabular}{c | c}
  		\textit{key} & \textit {value} \\\hline
  		to & 7 \\
  		tea & 3 \\
  		ted & 4 \\
  		ten & 12 \\
  		A & 15 \\
  		i & 11 \\
  		in & 5 \\
  		inn & 9 \\
  	\end{tabular}

  \end{column}
  \begin{column}{0.80\textwidth}
    \relscale{0.8}
    \begin{figure}
      \centering
      \begin{forest}
      for tree={align=center, inner sep=0pt, text centered, circle, font=\sffamily\bfseries, draw=black, fill=white, text width=1.5em }
      [\_, name=ROOT
        [\_, name=ROOT_T, edge label={node[midway,left] {\small{t}}}
          [7, name=ROOT_TO, edge label={node[midway,left] {\small{o}}}]
          [\_, name=ROOT_TE, edge label={node[midway,left] {\small{e}}}
            [3, edge label={node[midway,left] {\small{a}}}]
            [4, edge label={node[midway,left] {\small{d}}}]
            [12, edge label={node[midway,left] {\small{n}}}]
          ]
        ]
        [15, name=ROOT_A, edge label={node[midway,right] {\small{A}}}]
        [11, name= ROOT_i, edge label={node[midway,right] {\small{i}}}
          [5, edge label={node[midway,left] {\small{n}}}
            [9, edge label={node[midway,left] {\small{n}}}]
          ]
        ]
      ]
      \node[treenode, draw=red, left=0.5cm of ROOT, text = red] (nROOT) {\_};
      \node[treenode, draw=red, left=0.5cm of ROOT_T, text = red] (nROOT_T) {\_};
      \node[treenode, draw=red, left=0.5cm of ROOT_TO, text = red] (nROOT_TO) {7};
      \node[treenode, draw=red, below left=0.5cm of nROOT_TO, text = red] (nROOT_TON) {\_};
      \node[treenode, draw=red, below left=0.5cm of nROOT_TON, text = red] (nROOT_TONE) {10};
      \draw [->,red] (nROOT) -- (ROOT_A);
      \draw [->,red] (nROOT) -- (ROOT_i);
      \draw [->,red] (nROOT_T) -- (ROOT_TE);
      \draw [->,red] (nROOT) -- (nROOT_T) node[midway,left] {\small{t}};
      \draw [->,red] (nROOT_T) -- (nROOT_TO) node[midway,left] {\small{o}};
      \draw [->,red] (nROOT_TO) -- (nROOT_TON) node[midway,left] {\small{n}};
      \draw [->,red] (nROOT_TON) -- (nROOT_TONE) node[midway,left] {\small{e}};
      %edge label={node[midway,left] {\small{Help!}}}
      \end{forest}
      \label{fig:trie1}
      \end{figure}
    
  \end{column}
\end{columns}

\end{frame}


\begin{frame}
  \centerline{Data.Map}
\end{frame}

\begin{frame}[fragile]
  \frametitle{Добавяне на ключ в Trie}
\begin{verbatim}
import qualified Data.Map as Map

fromList, lookup, findWithDefault, 
insert, insertWith, delete
null, size, member
\end{verbatim}
\end{frame}

\begin{frame}
  \centerline{Благодаря за вниманието!}
\end{frame}


\end{document}


\begin{columns}[t]
  \begin{column}{0.3\textwidth}

  \end{column}
  \begin{column}{0.7\textwidth}

  \end{column}
\end{columns}


