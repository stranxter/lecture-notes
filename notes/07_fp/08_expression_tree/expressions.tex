\documentclass{beamer}
\usepackage{relsize}
\usepackage{color}

\usepackage{listings}
\usetheme{CambridgeUS}
%\usepackage{beamerthemesplit} % new
\usepackage{enumitem}
\usepackage{amsmath}                    % See geometry.pdf to learn the layout options.
\usepackage{amsthm}                   % See geometry.pdf to learn the layout options. There
\usepackage{amssymb}                    % See geometry.pdf to learn the layout options.
\usepackage[utf8]{inputenc}
\usepackage{graphicx}
\usepackage[english,bulgarian]{babel}

\usepackage{caption}
\usepackage{tikz}
\usetikzlibrary{shapes,arrows,positioning,calc,positioning,fit,chains}


\usetheme{CambridgeUS}
\usecolortheme{crane}

\lstset{language=C++,
                basicstyle=\ttfamily,
                keywordstyle=\color{blue}\ttfamily,
                stringstyle=\color{red}\ttfamily,
                commentstyle=\color{green}\ttfamily,
                morecomment=[l][\color{magenta}]{\#}
}

\tikzset{
block/.style = {draw, fill=white, rectangle,align = center},
entry/.style = {draw, fill=black, circle, radius=3em},
condition/.style = {draw, fill=white, diamond, align = center,node distance=3cm},
fork/.style = {draw, fill=black, circle,inner sep=1pt},
lnode/.style={rectangle split, rectangle split parts=3,draw, rectangle split horizontal},
treenode/.style = {align=center, inner sep=0pt, text centered, circle, font=\sffamily\bfseries, draw=black, fill=white, text width=1.5em},
token/.style={rectangle split, rectangle split parts=2,draw, rectangle split horizontal=false}
}

\newtheorem{mydef}{Дефиниция}[section]
\newtheorem{lem}{Лема}[section]
\newtheorem{thm}{Твърдение}[section]

\DeclareMathOperator{\restrict}{\upharpoonright}

\setitemize{label=\usebeamerfont*{itemize item}%
  \usebeamercolor[fg]{itemize item}
  \usebeamertemplate{itemize item}}

\setbeamercovered{transparent}

\captionsetup{font=tiny} 

\begin{document}
\title[Дърво на израза]{Оценка на аритметичен израз}
\frame{\titlepage}

\begin{frame}
  \centerline{Съвсем малко теория}
\end{frame}
  
  
  \subsection{Формални езици}
  
  
  \begin{frame}[fragile]
  \frametitle{Формални граматики}
  
  \begin{verbatim}
  the cat meows.
  the dog barks at the cat.
  the student lies to the teacher.
  \end{verbatim}
  
  \pause
  
  \begin{itemize}
    \item Азбука: $\Sigma=\{a..z\}$
    \item Нетерминални символи: $\{\mathbf{Verb},\mathbf{Object},\mathbf{Subject}, \mathbf{Prep}, \mathbf{Sentence}\}$
    \item Продукционни правила:
  \end{itemize}
  
  \pause
  
  $\mathbf{Object} \rightarrow cat | dog | student$
  
  $\mathbf{Subject} \rightarrow cat | dog | teacher$
  
  $\mathbf{Verb} \rightarrow meows | barks | lies$
  
  $\mathbf{Prep} \rightarrow to | at$
  
  $\mathbf{Sentence} \rightarrow the  \quad \mathbf{Object} \quad  \mathbf{Verb}$
  
  $\mathbf{Sentence} \rightarrow the \quad \mathbf{Object}  \quad\mathbf{Verb}  \quad\mathbf{Prep} \quad the \quad \mathbf{Subject}$
  
  
  
  \end{frame}
  
  
  \begin{frame}[fragile]
  \frametitle{Извод на \texttt{the cat meows at the dog}}
  
  \relscale{0.9}
  
  \begin{flushright}
  
  \relscale{0.5}
  
  $\mathbf{Object} \rightarrow cat | dog | student$
  
  $\mathbf{Subject} \rightarrow cat | teacher$
  
  $\mathbf{Verb} \rightarrow meows | barks | lies$
  
  $\mathbf{Prep} \rightarrow to | at$
  
  $\mathbf{Sentence} \rightarrow the  \quad \mathbf{Object} \quad  \mathbf{Verb}$
  
  $\mathbf{Sentence} \rightarrow the \quad \mathbf{Object}  \quad\mathbf{Verb}  \quad\mathbf{Prep} \quad the \quad \mathbf{Subject}$
  
  \end{flushright}
  
  
  \vspace{0.2cm}
  \pause
  
  $\mathbf{Sentence} \rightarrow the \quad \mathbf{Object}  \quad\mathbf{Verb}  \quad\mathbf{Prep} \quad the \quad \mathbf{Subject}$
  
  \pause
  
  \begin{flushright}
  \relscale{0.5}
  $\mathbf{Object} \rightarrow cat$
  \end{flushright}
  
  $\mathbf{Sentence} \rightarrow the \quad cat \quad\mathbf{Verb}  \quad\mathbf{Prep} \quad the \quad \mathbf{Subject}$
  
  
  
  \pause
  
  \begin{flushright}
  \relscale{0.5}
  $\mathbf{Verb} \rightarrow meows$
  \end{flushright}
  
  $\mathbf{Sentence} \rightarrow the \quad cat \quad meows  \quad\mathbf{Prep} \quad the \quad \mathbf{Subject}$
  
  \pause
  
  \begin{flushright}
  \relscale{0.5}
  $\mathbf{Prep} \rightarrow at$
  \end{flushright}
  
  $\mathbf{Sentence} \rightarrow the \quad cat \quad meows  \quad at \quad the \quad \mathbf{Subject}$
  
  
  
  \pause
  
  \begin{flushright}
  \relscale{0.5}
  $\mathbf{Subject} \rightarrow dog$
  \end{flushright}
  
  $\mathbf{Sentence} \rightarrow the \quad cat \quad meows  \quad at \quad the \quad dog$
  
  
  \end{frame}
  
  
  \begin{frame}[fragile]
  \frametitle{Синтактично дърво}
  
  
  \centering
  \begin{tikzpicture}[auto, sibling distance=1.5cm,>=latex',level 2/.style={sibling distance=3cm}]
  \node [] {Sentence}
    child {
      node [style = {text=red}] {the}
    }
    child {
      node [] {Object}
      child {
        node [style = {text=red}] {teacher}
      }
    }
    child {
      node [] {Verb}
      child {
        node [style = {text=red}] {barks}
      }
    }
    child {
      node [] {Prep}
      child {
        node [style = {text=red}] {at}
      }
    }
    child {
      node [style = {text=red}] {the}
    }
    child {
      node [] {Subject}
      child {
        node [style = {text=red}] {student}
      }
    };
  \end{tikzpicture}
  
  \end{frame}
  
  \begin{frame}[fragile]
  \frametitle{Мета-език на Бекус-Наур}
  
\begin{verbatim}
<digit> ::= 0 | 1 | 2 | 3 | 4 | 5 | 6 | 7 | 8 | 9 
<unsigned int> ::= <digit>+
<integer> ::= [+|-] <unsigned int>
<identifier> ::= _ (<letter> | <digit> | _ )* 
<identifier> ::= <leter> (<letter> | <digit> | _ )* 
\end{verbatim}
  
\end{frame}


\subsection{Аритметичен израз}
\begin{frame}
  \centerline{Аритметичен израз}
\end{frame}


\begin{frame}[fragile]
\frametitle{Аритметичен израз}

\begin{small}
\begin{verbatim}
<expression> ::= <number> 
<expression> ::= (<expression> <operator> <exprerssion>)
<number> ::= {0,..,9}+
<operator> ::= + | - | * | /
\end{verbatim}
\end{small}

\end{frame}  


\begin{frame}
\centerline{Лексически анализ}
\end{frame}

\begin{frame}[fragile]
\frametitle{Лексически анализ}

\begin{verbatim}
  (1   +2)
  
      *    3
\end{verbatim}

\bigskip
\bigskip

\begin{tikzpicture}[auto, scale=0.6, every node/.style={scale=0.6}, node distance=3cm,>=latex']
\node[token] (n1) {\nodepart{two}(\nodepart{one}OPEN\_PAR};
\node[token, right of = n1] (n2) {\nodepart{two}1\nodepart{one}NUMBER};
\node[token, right of = n2] (n3) {\nodepart{two}+\nodepart{one}OPERATOR};
\node[token, right of = n3] (n4) {\nodepart{two}2\nodepart{one}NUMBER};
\node[token, right of = n4] (n5) {\nodepart{two})\nodepart{one}CLOSE\_PAR};
\node[token, right of = n5] (n6) {\nodepart{two}*\nodepart{one}OPERATOR};
\node[token, right of = n6] (n7) {\nodepart{two}3\nodepart{one}NUMBER};
\end{tikzpicture}

\bigskip
\bigskip

\begin{lstlisting}[basicstyle=\small,language=Haskell]
data Token = Number Int | 
             Operator Char | 
             OpenParen | 
             CloseParen

lexer :: String -> [String]
tokens :: String -> [Token]
\end{lstlisting}
  

\end{frame}

\begin{frame}[fragile]
  \frametitle{Синтактично дърво}


\begin{columns}[t]
  \begin{column}{0.3\textwidth}

    \begin{figure}
      \centering
      \begin{tikzpicture}[auto, node distance=2cm,>=latex']
      \node [treenode] {*}
        child {
          node [treenode] {+}
          child {
            node [treenode] {1}
          }
          child {
            node [treenode] {2}
          }
        }
        child {
          node [treenode] {3}
        };
      \end{tikzpicture}
      \caption{Дърво на израза \texttt{(1+2)*3}.}
      \label{fig:treeexpr}
      \end{figure}
      

  \end{column}
  \begin{column}{0.7\textwidth}

\begin{lstlisting}[basicstyle=\small,language=Haskell]
data Expression = 
     Constant Int | 
     Op Expression Char Expression

parse :: [Token] 
         -> Maybe (Expression,[Token])     
parsestr = (unfoldr parse) . tokens
\end{lstlisting}

  \end{column}
\end{columns}

\end{frame}  
  

\begin{frame}[fragile]
  \frametitle{Оценка на дърво}


\begin{columns}[t]
  \begin{column}{0.3\textwidth}

    \begin{figure}
      \centering
      \begin{tikzpicture}[auto, node distance=2cm,>=latex']
      \node [treenode] {*}
        child {
          node [treenode] {+}
          child {
            node [treenode] {1}
          }
          child {
            node [treenode] {2}
          }
        }
        child {
          node [treenode] {3}
        };
      \end{tikzpicture}
      \caption{Дърво на израза \texttt{(1+2)*3}.}
      \label{fig:treeexpr}
      \end{figure}
      

  \end{column}
  \begin{column}{0.7\textwidth}

\begin{lstlisting}[basicstyle=\small,language=Haskell]
eval :: Expression -> Int
eval (Constant n) = ...
eval (Op left op right) = ...
\end{lstlisting}

  \end{column}
\end{columns}

\end{frame}  


\begin{frame}[fragile]
  \frametitle{Добавяне на променливи}


\begin{columns}[t]
  \begin{column}{0.3\textwidth}

    \begin{figure}
      \centering
      \begin{tikzpicture}[auto, node distance=2cm,>=latex']
      \node [treenode] {*}
        child {
          node [treenode] {+}
          child {
            node [treenode] {1}
          }
          child [] {
            node [treenode,style={draw=red}] {x}
          }
        }
        child {
          node [treenode] {3}
        };
      \end{tikzpicture}
      \caption{Дърво на израза \texttt{(1+x)*3}.}
      \label{fig:treeexpr}
      \end{figure}
      

  \end{column}
  \begin{column}{0.7\textwidth}

\begin{lstlisting}[basicstyle=\small,language=Haskell]

data Token = Number Int | 
             Operator Char | 
             OpenParen | 
             CloseParen |
             Variable String

data Expression = 
     Constant Int | 
     Op Expression Char Expression |
     Var String
     

\end{lstlisting}

  \end{column}
\end{columns}

\end{frame}  


\begin{frame}[fragile]
  \frametitle{Оценка на променливи}


\begin{columns}[t]
  \begin{column}{0.3\textwidth}

    \begin{figure}
      \centering
      \begin{tikzpicture}[auto, node distance=2cm,>=latex']
      \node [treenode] {*}
        child {
          node [treenode] {+}
          child {
            node [treenode] {1}
          }
          child [] {
            node [treenode,style={draw=red}] {x}
          }
        }
        child {
          node [treenode] {3}
        };
      \end{tikzpicture}
      \caption{Дърво на израза \texttt{(1+x)*3}.}
      \label{fig:treeexpr}
      \end{figure}
      

  \end{column}
  \begin{column}{0.7\textwidth}

\begin{lstlisting}[basicstyle=\small,language=Haskell]

type Environment = [(String,Int)]
eval :: Environment -> Expression -> Int     

\end{lstlisting}

  \end{column}
\end{columns}

\end{frame}  

\begin{frame}
  \centerline{Благодаря за вниманието!}
\end{frame}


\end{document}


\begin{columns}[t]
  \begin{column}{0.2\textwidth}

\relscale{0.63}
\begin{lstlisting}
\end{lstlisting}
\relscale{1}

  \end{column}
  \begin{column}{0.8\textwidth}

  \end{column}
\end{columns}


