\documentclass{beamer}
\usepackage{relsize}
\usepackage{color}

\usepackage{listings}
\usetheme{CambridgeUS}
%\usepackage{beamerthemesplit} % new
\usepackage{enumitem}
\usepackage{amsmath}                    % See geometry.pdf to learn the layout options.
\usepackage{amsthm}                   % See geometry.pdf to learn the layout options. There
\usepackage{amssymb}                    % See geometry.pdf to learn the layout options.
\usepackage[utf8]{inputenc}
\usepackage{graphicx}
\usepackage[english,bulgarian]{babel}
\usepackage{tikz}


\usepackage{caption}

\usetheme{CambridgeUS}
\usecolortheme{crane}

\lstset{language=C++,
                basicstyle=\ttfamily,
                keywordstyle=\color{blue}\ttfamily,
                stringstyle=\color{red}\ttfamily,
                commentstyle=\color{green}\ttfamily,
                morecomment=[l][\color{magenta}]{\#}
}

\newtheorem{mydef}{Дефиниция}[section]
\newtheorem{lem}{Лема}[section]
\newtheorem{thm}{Твърдение}[section]

\DeclareMathOperator{\restrict}{\upharpoonright}

\setitemize{label=\usebeamerfont*{itemize item}%
  \usebeamercolor[fg]{itemize item}
  \usebeamertemplate{itemize item}}

\setbeamercovered{transparent}

\captionsetup{font=tiny} 

\begin{document}
\title[Типове]{Дефиниране на типове (1)}
\frame{\titlepage}

\section{Дефиниране на типове}
\subsection{data}


\begin{frame}[fragile]
  \frametitle{Изброим тип}

\begin{lstlisting}[basicstyle=\small]
  data Season = Winter | Spring | Summer | Autumn

  season month
    | month < 2 = Winter
    | month < 5 = Spring
    | month < 8 = Summer
    | month < 11 = Autumn
    | otherwise = Winter
\end{lstlisting}

\begin{itemize}
  \item Конструктор
\end{itemize}

\end{frame}

\begin{frame}[fragile]
  \frametitle{Някои неща ``на готово''}

\begin{itemize}
  \item Образци
  \begin{lstlisting}[basicstyle=\small]
      seasonn Spring = 1
      seasonn Summer = 2
      seasonn Autumn = 3
      seasonn Winter = 4
  \end{lstlisting}
  \item Сравнение
  \begin{lstlisting}[basicstyle=\small]
      ...deriving (Eq)
      >Winter == Winter
      ...deriving (Ord)
      >Winter < Spring
  \end{lstlisting}
  \item Сравнение
  \begin{lstlisting}[basicstyle=\small]
      ...deriving (Ord)
      >Winter < Spring
  \end{lstlisting}
\end{itemize}

\end{frame}

\begin{frame}[fragile]
  \frametitle{Някои неща ``на готово''}

\begin{itemize}
  \item Извеждане, въвеждане
  \begin{lstlisting}[basicstyle=\small]
      ...deriving (Show, Read)
  \end{lstlisting}
  \item Интервали
  \begin{lstlisting}[basicstyle=\small]
      ...deriving (Enum)
      >[Spring .. Winter]
  \end{lstlisting}
  \item Преобразуване към цяло число
  \begin{lstlisting}[basicstyle=\small]
      ...deriving (Enum)
      >fromEnum Summer
      >toEnum 2 :: Season
  \end{lstlisting}
\end{itemize}

\end{frame}

\begin{frame}[fragile]
  \frametitle{Конструктори с параметри}

  \begin{lstlisting}[basicstyle=\small]
    data Point = Point Double Double
  \end{lstlisting}

\begin{itemize}
  \item Защо не просто двойка $(Double, Double)$?
\end{itemize}

\end{frame}


\begin{frame}[fragile]
  \frametitle{Синоним на тип}

  \begin{lstlisting}[basicstyle=\small]
    data Point = (Double, Double)
  \end{lstlisting}

\begin{itemize}
  \item Може и така... но!
\end{itemize}

\end{frame}


\begin{frame}[fragile]
  \frametitle{Полиморфизъм}

  \begin{lstlisting}[basicstyle=\small]
    data Shape = Circle Double | Rectangle Double Double

    area :: Shape -> Double
    area (Circle r) = pi * r * r
    area (Rectangle a b) = a * b
  \end{lstlisting}

\begin{itemize}
  \item Предефиниране на функции (overloading)
\end{itemize}

\end{frame}

\begin{frame}[fragile]
  \frametitle{Какво всъщност са конструкторите?}

  \begin{lstlisting}[basicstyle=\small]
    >:t Circle
  \end{lstlisting}

\end{frame}

\begin{frame}[fragile]
  \frametitle{Записи}

  \begin{lstlisting}[basicstyle=\small]
    data Student = Student {name :: String
                           ,fn :: String
                           ,grades :: [(String,Double)]}

    x :: Student = Student {name = "Ivan"
                           ,fn = "123456"
                           ,grades = [("OOP",5.5),
                                      ("FP",6.0)]}                           

    >grades x
    >:t grades
  \end{lstlisting}

\end{frame}

\begin{frame}[fragile]
    \frametitle{}
  
  \centerline{Благодаря за вниманието!}
\newcommand{\license}[1]{
  \begin{tikzpicture}[remember picture,overlay]
      \node[xshift=0mm,yshift=13mm,anchor=south east] at (current page.south east)
      {\tiny{Материалите са разработени от Калин Георгиев за курсовете по програмиране на ФМИ, СУ}};
      \node[xshift=0mm,yshift=10mm,anchor=south east] at (current page.south east)
      {\tiny{Creative Commons Attribution-NonCommercial-ShareAlike 4.0 International}};
      \node[xshift=0mm,yshift=5mm,anchor=south east] at (current page.south east){%
      \includegraphics[width=30mm]{{#1}/license}};
    \end{tikzpicture}
}
\license{../../..}
 
\end{frame}  

\end{document}



\begin{columns}[t]
  \begin{column}{0.2\textwidth}

\relscale{0.63}
\begin{lstlisting}
\end{lstlisting}
\relscale{1}

  \end{column}
  \begin{column}{0.8\textwidth}

  \end{column}
\end{columns}


