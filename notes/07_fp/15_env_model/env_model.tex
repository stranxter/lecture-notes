\documentclass{beamer}
\usepackage{relsize}
\usepackage{color}

\usepackage{listings}
\usetheme{CambridgeUS}
%\usepackage{beamerthemesplit} % new
\usepackage{enumitem}
\usepackage{amsmath}                    % See geometry.pdf to learn the layout options.
\usepackage{amsthm}                   % See geometry.pdf to learn the layout options. There
\usepackage{amssymb}                    % See geometry.pdf to learn the layout options.
\usepackage[utf8]{inputenc}
\usepackage{graphicx}
\usepackage[english,bulgarian]{babel}

\usepackage{caption}
\usepackage{tikz}
\usepackage{forest}

\usetikzlibrary{shapes,arrows,positioning,calc,positioning,fit,chains,3d,arrows.meta, calc}

\usetheme{CambridgeUS}
\usecolortheme{crane}

\lstset{language=C++,
                basicstyle=\ttfamily,
                keywordstyle=\color{blue}\ttfamily,
                stringstyle=\color{red}\ttfamily,
                commentstyle=\color{green}\ttfamily,
                morecomment=[l][\color{magenta}]{\#}
}

\tikzset{
block/.style = {draw, fill=white, rectangle,align = center},
entry/.style = {draw, fill=black, circle, radius=3em},
condition/.style = {draw, fill=white, diamond, align = center,node distance=3cm},
fork/.style = {draw, fill=black, circle,inner sep=1pt},
lnode/.style={rectangle split, rectangle split parts=3,draw, rectangle split horizontal},
treenode/.style = {align=center, inner sep=0pt, text centered, circle, font=\sffamily\bfseries, draw=black, fill=white, text width=1.5em},
flexnode/.style = {align=center, text centered, ellipse, font=\sffamily\bfseries, draw=black, fill=white},
token/.style={rectangle split, rectangle split parts=2,draw, rectangle split horizontal=false},
box/.style={
        draw, 
        rounded corners, 
        fill=gray!15, 
        align=left, 
        inner sep=10pt
    },
    circle node/.style={
        draw, 
        circle, 
        inner sep=0pt, 
        minimum size=0.8cm,
        font=\large
    },
    arrow/.style={
        -Stealth, 
        thick,
        rounded corners=2pt
    }
}

\usepackage{color}
\definecolor{lightgray}{rgb}{.9,.9,.9}
\definecolor{darkgray}{rgb}{.4,.4,.4}
\definecolor{purple}{rgb}{0.65, 0.12, 0.82}

\lstdefinelanguage{JavaScript}{
  keywords={typeof, new, true, false, catch, function, return, null, catch, switch, var, if, in, while, do, else, case, break},
  keywordstyle=\color{blue}\bfseries,
  ndkeywords={class, export, boolean, throw, implements, import, this},
  ndkeywordstyle=\color{darkgray}\bfseries,
  identifierstyle=\color{black},
  sensitive=false,
  comment=[l]{//},
  morecomment=[s]{/*}{*/},
  commentstyle=\color{purple}\ttfamily,
  stringstyle=\color{red}\ttfamily,
  morestring=[b]',
  morestring=[b]"
}

\newtheorem{mydef}{Дефиниция}[section]
\newtheorem{lem}{Лема}[section]
\newtheorem{thm}{Твърдение}[section]

\DeclareMathOperator{\restrict}{\upharpoonright}

\setitemize{label=\usebeamerfont*{itemize item}%
  \usebeamercolor[fg]{itemize item}
  \usebeamertemplate{itemize item}}

\setbeamercovered{transparent}

\captionsetup{font=tiny} 

\begin{document}
\title[Scheme]{Модел на средите}
\frame{\titlepage}

\section{Модел на средите}

\begin{frame}[fragile]
\frametitle{Модел на средите}

\begin{lstlisting}[language=Lisp]
(define x 5)
(define y 0)
(define (plusx y) (+x y))
(plusx 10)
\end{lstlisting}


\begin{tikzpicture}[scale=0.5,
    transform shape]
  % --- Global Environment ---
    \node[box, minimum width=9cm, minimum height=2.5cm] (global) at (0,0) {};
    
    % Text inside Global Environment
    \node[anchor=north west] at ($(global.north west)+(0,-0.2)$) {
        \textbf{\textit{x:5}}
    };
    \node[anchor=north west] at ($(global.north west)+(0,-0.7)$) {
        \textbf{\textit{y:0}}
    };
    \node[anchor=west] (square_text) at ($(global.north west)+(0, -1.4)$) {
        \textbf{plusx:}
    };

    % Label for Global Environment
    \node[left=1.5cm of global.north west, anchor=north east, align=left, yshift=-0.3cm] (global_label) {
        global\\env
    };
    \draw[arrow] (global_label.east) -- (global.west |- global_label.east);


    % --- Function Object (Closure) ---
    % Place the two circles
    \node[circle node, below=2.5cm of square_text, xshift=1cm] (car) {$\bullet$};
    \node[circle node, right=-0.4pt of car] (cdr) {$\bullet$}; % Slightly overlap to look connected

    % Pointer from "square:" to the closure
    \draw[arrow] (square_text.south) ++(0,0) |- (car.west);

    % Function details (parameters and body)
    \node[below=1cm of car, anchor=north, align=left] (func_details) {
        parameters: x\\[0.2em]
        body: (+ x y)
    };
    \draw[arrow] (car.south) -- (func_details.north);

    % Pointer from closure back to global env
    \draw[arrow] (cdr.north)  -- (cdr.north |- global.south);


    % --- Execution Frame E1 ---
    \node[box, minimum width=2.5cm, right=3cm of cdr, anchor=west] (e1) {
        \textbf{y: 10}
    };

    % Label for E1
    \node[left=0.8cm of e1] (e1_label) {E1};
    \draw[arrow] (e1_label) -- (e1);

    % Pointer from E1 back to global env
    \draw[arrow] (e1.north) -- (e1.north |- global.south);

    % The body expression being evaluated
    \node[below=0.8cm of e1] {
        (\textbf{+} x y)
    };

\end{tikzpicture}

\end{frame}


\begin{frame}[fragile]
\frametitle{Модел на средите}

\begin{lstlisting}[language=Lisp]
(define (fact x) (if (< x 2) 
                     1 
                     (* x (fact (- x 1)))))
(fact 2)
\end{lstlisting}


\begin{tikzpicture}[scale=0.5,
  transform shape]
  % --- Global Environment ---
  \node[box, minimum width=9cm, minimum height=2.5cm] (global) at (0,0) {};
  
  % Text inside Global Environment
  \node[anchor=west] (fact_text) at ($(global.north west)+(0.5, -1.4)$) {
    \textbf{fact:}
  };

  % Label for Global Environment
  \node[left=1.5cm of global.north west, anchor=north east, align=left, yshift=-0.3cm] (global_label) {
    global\\env
  };
  \draw[arrow] (global_label.east) -- (global.west |- global_label.east);


  % --- Function Object (Closure) ---
  % Place the two circles
  \node[circle node, below=2.5cm of fact_text, xshift=1cm] (car) {$\bullet$};
  \node[circle node, right=-0.4pt of car] (cdr) {$\bullet$};

  % Pointer from "fact:" to the closure
  \draw[arrow] (fact_text.south) ++(0,0) |- (car.west);

  % Function details (parameters and body)
  \node[below=1cm of car, anchor=north, align=left] (func_details) {
    parameters: x\\[0.2em]      
    body: (if (< x 2)\\
    \phantom{body: (}1\\
    \phantom{body: (}x (fact (- x 1))))\\
  };
  \draw[arrow] (car.south) -- (func_details.north);

  % Pointer from closure back to global env
  \draw[arrow] (cdr.north)  -- (cdr.north |- global.south);


  % --- Execution Frame E1 ---
  \node[box, minimum width=2.5cm, right=3cm of cdr, anchor=west] (e1) {
    \textbf{x: 2}
  };

  % Label for E1
  \node[left=0.8cm of e1] (e1_label) {E1};
  \draw[arrow] (e1_label) -- (e1);

  % Pointer from E1 back to global env
  \draw[arrow] (e1.north) -- (e1.north |- global.south);


  % --- Execution Frame E2 ---
  \node[box, minimum width=2.5cm, below=1.5cm of e1] (e2) {
    \textbf{x: 1}
  };

  % Label for E2
  \node[left=0.8cm of e2] (e2_label) {E2};
  \draw[arrow] (e2_label) -- (e2);

  % Pointer from E2 back to E1
  \draw[arrow] (e2.north) -- (e2.north |- e1.south);


  % The body expression being evaluated in E1
    \node[right=0.5cm of e1.east] {
      (if (< 2 2) (2 * (fact 1))...)
    };

  % The body expression being evaluated in E2
  \node[right=0.5cm of e2] {
    (if (< 1 2) ... 1)
  };

\end{tikzpicture}

\end{frame}


\begin{frame}[fragile]
\frametitle{Модел на средите}

\begin{lstlisting}[language=Lisp]
(define inc
  (let ((counter 0)) 
       (lambda (x) 
          (begin (set! counter (+ counter x)) 
                 counter))))
(inc 1)                 
(inc 2)
\end{lstlisting}


\begin{tikzpicture}[scale=0.5,
    transform shape]
  % --- Global Environment ---
    \node[box, minimum width=9cm, minimum height=1.5cm] (global) at (0,0) {};
    
    % Text inside Global Environment
    \node[anchor=west] (square_text) at ($(global.north west)+(0, -0.4)$) {
        \textbf{inc:}
    };

    % Label for Global Environment
    \node[left=1.5cm of global.north west, anchor=north east, align=left, yshift=-0.3cm] (global_label) {
        global\\env
    };
    \draw[arrow] (global_label.east) -- (global.west |- global_label.east);


    % --- Function Object (Closure) ---
    % Place the two circles
    \node[circle node, below=2.5cm of square_text, xshift=1cm] (car) {$\bullet$};
    \node[circle node, right=-0.4pt of car] (cdr) {$\bullet$}; % Slightly overlap to look connected

    % Pointer from "square:" to the closure
    \draw[arrow] (square_text.south) ++(0,0) |- (car.west);

    % Function details (parameters and body)
    \node[below=1cm of car, anchor=north, align=left] (func_details) {
        parameters: x\\[0.2em]
        body: (begin (set! counter (+ counter x)) \\
\phantom{body: (}counter))))
    };
    \draw[arrow] (car.south) -- (func_details.north);


    % --- Execution Frame E1 ---
    \node[box, minimum width=2.5cm, right=3cm of cdr] (e1) {
        \textbf{counter: 0}
    };

    % Pointer from closure 
    \draw[arrow] (cdr.east)  -- ($(e1.west)+(0,0)$);

    % Label for E1
    \node[right=0.8cm of e1] (e1_label) {E1};
    \draw[arrow] (e1_label) -- (e1);

    % Pointer from E1 back to global env
    \draw[arrow] (e1.north) -- (e1.north |- global.south);

    
\end{tikzpicture}

\end{frame}

\section{Factory Function}

\begin{frame}
  \centerline{Factory Function. Closure Pattern}
\end{frame}

\begin{frame}[fragile]
\frametitle{Factory Function}

\begin{lstlisting}[basicstyle=\tiny,language=Lisp]
(define (counter init)
  (let ((count init)) 
       (lambda (step) 
          (begin (set! count (+ count step)) 
                 count))))
(define i1 (counter 0)) (define i2 (counter 100))           
(i1 1) (i1 1) (i1 1) (i1 1)
(i2 5) (i2 5) (i2 5)
\end{lstlisting}


\begin{tikzpicture}[scale=0.5,
    transform shape]
  % --- Global Environment ---
    \node[box, minimum width=18cm, minimum height=1cm] (global) at (0,0) {};
    
    % Text inside Global Environment
    \node[anchor=west] (square_text) at ($(global.north west)+(0, -0.4)$) {
        \textbf{i1:}
    };

    % Text inside Global Environment
    \node[anchor=west] (square_text2) at ($(global.north west)+(9, -0.4)$) {
        \textbf{i2:}
    };

    % Label for Global Environment
    \node[left=1.5cm of global.north west, anchor=north east, align=left, yshift=-0.3cm] (global_label) {
        global\\env
    };
    \draw[arrow] (global_label.east) -- (global.west |- global_label.east);


    % --- Function Object (Closure) ---
    % Place the two circles
    \node[circle node, below=3cm of square_text, xshift=1cm] (car) {$\bullet$};
    \node[circle node, right=-0.4pt of car] (cdr) {$\bullet$}; % Slightly overlap to look connected

    % Place the two circles
    \node[circle node, below=3cm of square_text, xshift=10cm] (car2) {$\bullet$};
    \node[circle node, right=-0.4pt of car2] (cdr2) {$\bullet$}; % Slightly overlap to look connected


    % Pointer from "square:" to the closure
    \draw[arrow] (square_text.south) ++(0,0) |- (car.west);
    % Pointer from "square:" to the closure
    \draw[arrow] (square_text2.south) ++(0,0) |- (car2.west);

    % Function details (parameters and body)
    \node[below=0.6cm of car, anchor=north, align=left,xshift=-0.8cm,fill=gray!25] (func_details) {
        parameters: step\\[0.2em]
        body: (begin (set! count \\
\phantom{body: (begin (set! }(+ count step)) \\
\phantom{body: (begin (}count))))
    };
    \draw[arrow] (car.south) -- (car.south |- func_details.north);


    % --- Execution Frame E1 ---
    \node[box, minimum width=2.5cm, right=1.5cm of cdr, anchor=west] (e1) {
        \textbf{count: 0}
    };

    % --- Execution Frame E0 ---
    \node[box, minimum width=2.5cm, above=0.7cm of e1, anchor=south] (e0) {
        \textbf{init: 0}
    };

    % Label for E0
    \node[right=0.8cm of e0] (e0_label) {E0};
    \draw[arrow] (e0_label) -- (e0);

    % Pointer from E0 back to global
    \draw[arrow] (e0.north) -- (e0.north |- global.south);

    % Pointer from closure back to global env
    \draw[arrow] (cdr.east)  -- ($(e1.west)+(0,0)$);

    % Label for E1
    \node[right=0.8cm of e1] (e1_label) {E1};
    \draw[arrow] (e1_label) -- (e1);

    % Pointer from E1 back to E0
    \draw[arrow] (e1.north) -- (e0.south);


    % --- Execution Frame E3 ---
    \node[box, minimum width=2.5cm, right=1.5cm of cdr2, anchor=west] (e3) {
        \textbf{count: 100}
    };

    % --- Execution Frame E2 ---
    \node[box, minimum width=2.5cm, above=0.7cm of e3,anchor=south] (e2) {
        \textbf{init: 100}
    };

    % Label for E2
    \node[right=0.8cm of e2] (e2_label) {E2};
    \draw[arrow] (e2_label) -- (e2);


    % Pointer from closure back to global env
    \draw[arrow] (cdr2.east)  -- ($(e3.west)+(0,0)$);

    % Label for E3
    \node[right=0.8cm of e3] (e3_label) {E3};
    \draw[arrow] (e3_label) -- (e3);

    % Pointer from E3 back to E2
    \draw[arrow] (e3.north) -- (e2.south);

    % Pointer from E2 back to global
    \draw[arrow] (e2.north) -- (e2.north |- global.south);

    % --- Execution Frame E2 ---
    \node[box, minimum width=2.5cm, above=0.7cm of e3,anchor=south] (e2) {
        \textbf{init: 100}
    };

    % --- Execution Frame E4 ---
    \node[box, minimum width=2.5cm, below=0.7cm of e1,anchor=north] (e4) {
        \textbf{step: 1}
    };

    % Label for E4
    \node[right=0.8cm of e4] (e4_label) {E4};
    \draw[arrow] (e4_label) -- (e4);

    % Pointer from E4 back to E1
    \draw[arrow] (e4.north) -- (e1.south);


    % --- Execution Frame E5 ---
    \node[box, minimum width=2.5cm, below=0.7cm of e3,anchor=north] (e5) {
        \textbf{step: 5}
    };

    % Label for E5
    \node[right=0.8cm of e5] (e5_label) {E5};
    \draw[arrow] (e5_label) -- (e5);

    % Pointer from E5 back to E3
    \draw[arrow] (e5.north) -- (e3.south);

    % Function body
    \node[below=0.9cm of e4, anchor=north, align=left] (func_details2) {
    (set! count (+ count step)) 
    };
    \draw[arrow] (func_details2.north) -- (e4.south);

    % Function body
    \node[below=0.9cm of e5, anchor=north, align=left] (func_details3) {
    (set! count (+ count step)) 
    };
    \draw[arrow] (func_details3.north) -- (e5.south);

\end{tikzpicture}

\end{frame}

\begin{frame}[fragile]
\frametitle{Factory Function в JavaScript}

\begin{lstlisting}[basicstyle=\tiny,language=JavaScript]
function counter(init) {

  let count = init;

  return {
    inc: step => count+=step,
    get: ()=>count
  };
}

const i1 = counter(0), i2 = counter(100);

i1.inc(1);i1.inc(1);i1.inc(1);i1.inc(1);
console.log(i1.get());

i2.inc(5);i2.inc(5);i2.inc(5);
console.log(i2.get());
\end{lstlisting}

\end{frame}

\begin{frame}[fragile]
\frametitle{Аналог в C++}

\begin{lstlisting}[basicstyle=\tiny,language=C++]
std::function<int(int)> counter(int init)
{
    int count = init;
    return [count](int step) mutable ->int{count+=step; return count;};
}

int main()
{
    std::function<int(int)> c1 = counter(0),
                            c2 = counter(100);

    c1(1);c1(1);c1(1);
    std::cout<<c1(1)<<std::endl;
    c2(5);c2(5);
    std::cout<<c2(5)<<std::endl;
}
\end{lstlisting}

\end{frame}

\begin{frame}[fragile]
\frametitle{Аналог в C++. Ad hoc полиморфизъм без класове?}

\begin{lstlisting}[basicstyle=\tiny,language=C++]
struct Counter
{
  std::function<int(int)> inc;
  std::function<int()> get;
};
std::function<Counter()> counter(int init)
{
  return [count = init]() mutable -> Counter 
        {
           return { [&count](int step) -> int { count += step; return count; },
                    [&count]() -> int { return count; } };
  };
}
int main()
{
    std::function<Counter()> c1 = counter(0),
                             c2 = counter(100);

    c1().inc(1);c1().inc(1);c1().inc(1);c1().inc(1);
    std::cout<<c1().get()<<std::endl;
    c2().inc(5);c2().inc(5);c2().inc(5);
    std::cout<<c2().get()<<std::endl;
}\end{lstlisting}

\end{frame}


\begin{frame}[fragile]
    \frametitle{}
  
  \centerline{Благодаря за вниманието!}
\newcommand{\license}[1]{
  \begin{tikzpicture}[remember picture,overlay]
      \node[xshift=0mm,yshift=13mm,anchor=south east] at (current page.south east)
      {\tiny{Материалите са разработени от Калин Георгиев за курсовете по програмиране на ФМИ, СУ}};
      \node[xshift=0mm,yshift=10mm,anchor=south east] at (current page.south east)
      {\tiny{Creative Commons Attribution-NonCommercial-ShareAlike 4.0 International}};
      \node[xshift=0mm,yshift=5mm,anchor=south east] at (current page.south east){%
      \includegraphics[width=30mm]{{#1}/license}};
    \end{tikzpicture}
}
\license{../../..}
 
\end{frame}  

\end{document}
 

\begin{columns}[t]
  \begin{column}{0.3\textwidth}

  \end{column}
  \begin{column}{0.7\textwidth}

  \end{column} 
\end{columns}


\begin{lstlisting}[basicstyle=\small,language=Haskell]
\end{lstlisting}


