\documentclass{beamer}
\usepackage{relsize}
\usepackage{color}

\usepackage{listings}
\usetheme{CambridgeUS}
%\usepackage{beamerthemesplit} % new
\usepackage{enumitem}
\usepackage{amsmath}                    % See geometry.pdf to learn the layout options.
\usepackage{amsthm}                   % See geometry.pdf to learn the layout options. There
\usepackage{amssymb}                    % See geometry.pdf to learn the layout options.
\usepackage[utf8]{inputenc}
\usepackage{graphicx}
\usepackage[english,bulgarian]{babel}

\usepackage{caption}
\usepackage{tikz}

\usetheme{CambridgeUS}
\usecolortheme{crane}

\lstset{language=C++,
                basicstyle=\ttfamily,
                keywordstyle=\color{blue}\ttfamily,
                stringstyle=\color{red}\ttfamily,
                commentstyle=\color{green}\ttfamily,
                morecomment=[l][\color{magenta}]{\#}
}

\newtheorem{mydef}{Дефиниция}[section]
\newtheorem{lem}{Лема}[section]
\newtheorem{thm}{Твърдение}[section]

\DeclareMathOperator{\restrict}{\upharpoonright}

\setitemize{label=\usebeamerfont*{itemize item}%
  \usebeamercolor[fg]{itemize item}
  \usebeamertemplate{itemize item}}

\setbeamercovered{transparent}

\captionsetup{font=tiny} 

\begin{document}
\title[Индуктивни СД]{Хетерогенни и йерархични структури. \\ Индуктивни структури от данни}
\frame{\titlepage}

\section{Хетерогенни списъци}
\subsection{}

\begin{frame}[fragile]
\frametitle{Хетерогенни списъци}

\begin{itemize}
  \item Списъците са от тип \verb#[a]#, следователно елементите са еднотипни
  \item Haskell позволява обеденение на типове
\end{itemize}

\bigskip

\begin{lstlisting}[basicstyle=\small,language=Haskell]
type Point =  (Float,Float)
data Figure = Circle Point Float | 
              Square Point Float |
              Rectangle Point Float Float |
              Triangle  Point Point Point 
\end{lstlisting}
\end{frame}

\begin{frame}[fragile]
  \frametitle{Хетерогенни списъци}
  
  \begin{lstlisting}[basicstyle=\small,language=Haskell]
girl :: [Figure] = [Triangle (200,200) (100,300) (300,300),
                    Circle (200,100) 50,
                    Rectangle (100,320) 20 100,
                    Rectangle (280,320) 20 100]
  \end{lstlisting}
\end{frame}

\section{Полиморфизъм}

\begin{frame}[fragile]
  \frametitle{Полиморфизъм}
  
  \begin{lstlisting}[basicstyle=\small,language=Haskell]
distance :: Point -> Point -> Float
distance (x1,y1) (x2,y2) = sqrt ((x1-x2)^2 + (y1-y2)^2)

perimeter :: Figure -> Float
perimeter (Circle _ r) = 2 * pi * r
perimeter (Square _ s) = 4 * s
perimeter (Rectangle _ w h) = 2 * (w + h)
perimeter (Triangle a b c) = 
              sum $ zipWith distance [a,b,c] [b,c,a]

ghci>sum $ map perimeter girl              
  \end{lstlisting}
\end{frame}

\section{SVG}

\begin{frame}
  \centerline{Леко отклонение: форматът SVG}
\end{frame}


\begin{frame}[fragile]
  \frametitle{Scalable Vector Graphics}
  
\begin{lstlisting}[basicstyle=\tiny,language=xml]
<svg xmlns="http://www.w3.org/2000/svg" version="1.1">
  <circle cx="200.0" cy="100.0" r="50.0" 
          fill="none" stroke="black"/>
  <rect x="100.0" y="320.0" width="20.0" height="100.0" 
        fill="none" stroke="black"/>
  <polygon points="200.0,200.0 100.0,300.0 300.0,300.0 " 
           fill="none" stroke="black"/>
</svg>  
\end{lstlisting}

\bigskip

\begin{lstlisting}[basicstyle=\tiny,language=haskell]
svg :: Figure -> String
svg (Circle (x,y) r) = "<circle cx=\"" ++ show x ++ "\"" ++
                               "cy=\"" ++ show y ++ "\"" ++ 
                               "r=\"" ++ show r ++ "\"" ++ 
                               "fill=\"none\" stroke=\"black\"/>\n"
\end{lstlisting}

\end{frame}

\section{Индуктивни типове}

\begin{frame}
  \centerline{Групата от фигури също е фигура}
\end{frame}

\begin{frame}[fragile]
  \frametitle{Индуктивен тип}
\begin{lstlisting}[basicstyle=\tiny,language=Haskell]
data Figure = Circle Point Float | 
              Square Point Float |
              Rectangle Point Float Float |
              Triangle  Point Point Point |
              Group [Figure]

girl :: Figure =   Group [Triangle (200,200) (100,300) (300,300),
                          Circle (200,100) 50,
                          Rectangle (100,320) 20 100,
                          Rectangle (280,320) 20 100]

boy :: Figure  =   Group [Rectangle (400,170)  100 140, 
                          Circle (450,100) 50,
                          Rectangle (400,320) 20 100,
                          Rectangle (480,320) 20 100]

picture :: Figure = Group [girl, boy, Rectangle (50,30) 500 500]
\end{lstlisting}

\begin{itemize}
  \item И това, разбира се, поражда рекурсия:
\end{itemize}

\begin{lstlisting}[basicstyle=\small,language=Haskell]
perimeter (Group figs) = sum $ map perimeter figs
\end{lstlisting}


\end{frame}



\begin{frame}
  \centerline{Благодаря за вниманието!}
\end{frame}


\end{document}


\begin{columns}[t]
  \begin{column}{0.2\textwidth}

\relscale{0.63}
\begin{lstlisting}
\end{lstlisting}
\relscale{1}

  \end{column}
  \begin{column}{0.8\textwidth}

  \end{column}
\end{columns}


