\documentclass{beamer}
\usepackage{relsize}
\usepackage{color}

\usepackage{listings}
\usetheme{CambridgeUS}
%\usepackage{beamerthemesplit} % new
\usepackage{enumitem}
\usepackage{amsmath}                    % See geometry.pdf to learn the layout options.
\usepackage{amsthm}                   % See geometry.pdf to learn the layout options. There
\usepackage{amssymb}                    % See geometry.pdf to learn the layout options.
\usepackage[utf8]{inputenc}
\usepackage{graphicx}
\usepackage[english,bulgarian]{babel}

\usepackage{caption}
\usepackage{tikz}
\usepackage{forest}

\usetikzlibrary{shapes,arrows,positioning,calc,positioning,fit,chains}

\usetheme{CambridgeUS}
\usecolortheme{crane}

\lstset{language=C++,
                basicstyle=\ttfamily,
                keywordstyle=\color{blue}\ttfamily,
                stringstyle=\color{red}\ttfamily,
                commentstyle=\color{green}\ttfamily,
                morecomment=[l][\color{magenta}]{\#}
}

\tikzset{
block/.style = {draw, fill=white, rectangle,align = center},
entry/.style = {draw, fill=black, circle, radius=3em},
condition/.style = {draw, fill=white, diamond, align = center,node distance=3cm},
fork/.style = {draw, fill=black, circle,inner sep=1pt},
lnode/.style={rectangle split, rectangle split parts=3,draw, rectangle split horizontal},
treenode/.style = {align=center, inner sep=0pt, text centered, circle, font=\sffamily\bfseries, draw=black, fill=white, text width=1.5em},
flexnode/.style = {align=center, text centered, ellipse, font=\sffamily\bfseries, draw=black, fill=white},
token/.style={rectangle split, rectangle split parts=2,draw, rectangle split horizontal=false}
}


\newtheorem{mydef}{Дефиниция}[section]
\newtheorem{lem}{Лема}[section]
\newtheorem{thm}{Твърдение}[section]

\DeclareMathOperator{\restrict}{\upharpoonright}

\setitemize{label=\usebeamerfont*{itemize item}%
  \usebeamercolor[fg]{itemize item}
  \usebeamertemplate{itemize item}}

\setbeamercovered{transparent}

\captionsetup{font=tiny} 

\begin{document}
\title[Scheme]{Език за програмиране Scheme}
\frame{\titlepage}

\section{Основи}

\begin{frame}
  \centerline{Основни елементи на Scheme}
\end{frame}


\subsection{}

\begin{frame}[fragile]
\frametitle{Литерали, Символи, Атоми, S-изрази, комбинатори}

\begin{lstlisting}[basicstyle=\small\ttfamily,language=Lisp]
  5
  #f
  +
  (+ 1 2 3)
  (quotient 10 3)
  (remainder 10 3)
  (if #f 1 2)
  (+ (if 0 1 2) 1)
\end{lstlisting}

\end{frame}


\begin{frame}[fragile]
\frametitle{Дефиниции, функции и $\lambda$ функции}

\begin{lstlisting}[language=Lisp]
(define x 5)
(define (fact x) 
        (if (> x 1) 
            (* x (fact (- x 1))) 
            1))
((lambda (x) (+ x 1)) 0)
(define inc (lambda (x) (+ x 1)))
\end{lstlisting}
\begin{itemize}
  \item Предикати
\end{itemize}

\begin{lstlisting}[language=Lisp]
  even? odd?
\end{lstlisting}

\end{frame}


\begin{frame}[fragile]
\frametitle{Точкови двойки}

\begin{lstlisting}[language=Lisp]
(cons 1 2)
(define x (cons 1 (cons (2 (cons 3 0)))))
(car x)
(cdr x)
\end{lstlisting}
\begin{itemize}
  \item $c\{a,d\}^+r$ (до 4 влагания)
\end{itemize}

\end{frame}

\begin{frame}[fragile]
\frametitle{Цитирания}

\begin{itemize}
  \item Подтиска (отлага) прилагането на комбинатори и символи
  \item Но не и на литерали
\end{itemize}

\begin{lstlisting}[language=Lisp]
'1
(+ '1 '2)

'(+ 1 2)

('+ 1 2)
\end{lstlisting}

\end{frame}



\begin{frame}[fragile]
\frametitle{Списъци}

\begin{columns}[t]
  \begin{column}{0.7\textwidth}
    \begin{lstlisting}[language=Lisp]
    '()
    (cons 1 '())
    (cons 1 (cons 2 '()))

    (define l (list 1 2 3))
    (car l) (cdr l)
    (car (cdr l)) (cadr l)
  
    (define fl (list + inc fact))
    ((car fl) 1 2)

    (list 1 inc 2)
    (list 1 (list 2 3) 4)
  \end{lstlisting}

  \end{column}
  \begin{column}{0.3\textwidth}
    \begin{lstlisting}[language=Lisp]
      list
      null?
      length
      member
      append
    \end{lstlisting}
  \end{column} 
\end{columns}

\end{frame}


\begin{frame}[fragile]
\frametitle{Цитиране на S-изрази}

\begin{lstlisting}[language=Lisp]
(cdr '(+ 2 2))

(define expr '(+ 2 2))

(cons '- (cdr expr))

(eval (cons '- (cdr expr)))

((eval (car expr)) 1 2)
\end{lstlisting}

\end{frame}

\begin{frame}[fragile]
\frametitle{Функции от високо ниво за списъци}

\begin{lstlisting}[language=Lisp]
  (map (lambda (x) (+ x 1)) '(1 2 3 4))
  (filter even? '(1 2 3 4))
  (foldr + 0 '(1 2 3 4))
\end{lstlisting}

\end{frame}


\begin{frame}[fragile]
\frametitle{Безкрайни списъци (потоци)}

\begin{lstlisting}[language=Lisp]
(define ones (cons 1 (lambda() ones)))
(car ones) (car ((cdr ones)))

(define (ints i) (cons i (lambda () (ints (+ i 1)))))
(define allints (ints 0))
(car allints) (car ((crd allints)))
\end{lstlisting}

\end{frame}

\begin{frame}
  \centerline{Функции от високо ниво}
\end{frame}

\begin{frame}[fragile]
\frametitle{apply}
\begin{lstlisting}[language=Lisp]
(apply + '(1 2 3 4))

(apply + 1 2 '(3 4))
\end{lstlisting}
\end{frame}

\begin{frame}[fragile]
\frametitle{map}

\begin{itemize}
  \item $(map \; f \; l_1 \; l_2 \; ... \; l_n)$
\end{itemize}

\begin{lstlisting}[language=Lisp]
(map (lambda (x) (+ x 1)) '(1 2 3 4))
\end{lstlisting}
\begin{itemize}
  \item $f$ е едноаргументна функция
\end{itemize}

\begin{lstlisting}[language=Lisp]
(map + '(1 2 3) '(4 5 6) '(7 8 9))
\end{lstlisting}
\begin{itemize}
  \item $f$ е n-аргументна функция
\end{itemize}

\begin{lstlisting}[language=Lisp]
(apply map list '(1 2 3) '(4 5 6) '(7 8 9))
\end{lstlisting}

\end{frame}




\begin{frame}[fragile]
    \frametitle{}
  
  \centerline{Благодаря за вниманието!}
\newcommand{\license}[1]{
  \begin{tikzpicture}[remember picture,overlay]
      \node[xshift=0mm,yshift=13mm,anchor=south east] at (current page.south east)
      {\tiny{Материалите са разработени от Калин Георгиев за курсовете по програмиране на ФМИ, СУ}};
      \node[xshift=0mm,yshift=10mm,anchor=south east] at (current page.south east)
      {\tiny{Creative Commons Attribution-NonCommercial-ShareAlike 4.0 International}};
      \node[xshift=0mm,yshift=5mm,anchor=south east] at (current page.south east){%
      \includegraphics[width=30mm]{{#1}/license}};
    \end{tikzpicture}
}
\license{../../..}
 
\end{frame}  

\end{document}
 

\begin{columns}[t]
  \begin{column}{0.3\textwidth}

  \end{column}
  \begin{column}{0.7\textwidth}

  \end{column} 
\end{columns}


\begin{lstlisting}[basicstyle=\small,language=Haskell]
\end{lstlisting}

;let
;cond

